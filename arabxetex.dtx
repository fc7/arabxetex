% \iffalse
%
%!TEX encoding = UTF-8 Unicode
%
% Copyright © 2007-2010 by François Charette <firmicus at gmx dot net>
% 
% Distributable under the LaTeX Project Public License,
% version 1.3c or higher (your choice). The latest version of
% this license is at: http://www.latex-project.org/lppl.txt
%
% This work is "maintained" (as per LPPL maintenance status) 
% by François Charette.
% 
% This work consists of the file  arabxetex.dtx
%           and the derived files arabxetex.sty, arabxetex.pdf and
%           several accompanying *.map files.
%
%
%<*batchfile>
\begingroup
\input docstrip.tex
\keepsilent
\preamble

  ____________________________
  The arabxetex package         
  © 2007–2010  François Charette
  License information appended


\endpreamble
\postamble

Copyright © 2007–2010 by François Charette <firmicus at gmx dot net>

Distributable under the LaTeX Project Public License,
version 1.3c or higher (your choice). The latest version of
this license is at: http://www.latex-project.org/lppl.txt

This work is "maintained" (as per LPPL maintenance status) 
by François Charette.

This work consists of the file \jobname.dtx
and the derived files \jobname.sty and \jobname.pdf.
as well as an extensive collection of TECKit files 
(*.map, compiled as *.tec).

\endpostamble
\askforoverwritefalse
\generate{\file{\jobname.sty}{\from{\jobname.dtx}{package}}}
\generate{\file{\jobname.ins}{\from{\jobname.dtx}{batchfile}}}
\nopreamble\nopostamble
\generate{\file{README}{\from{\jobname.dtx}{readme}}}
\generate{\file{arabxetex-dtx-style.sty}{\from{\jobname.dtx}{dtx-style}}}
\endgroup
%</batchfile>
%
%<*driver>
\documentclass[11pt]{ltxdoc}
\usepackage{arabxetex-dtx-style}
\EnableCrossrefs
\CodelineIndex
\RecordChanges
%\OnlyDescription
\begin{document}
  \DocInput{\jobname.dtx}
\end{document}
%</driver>
%
%<*readme>
--------------------------
THE ARABXETEX PACKAGE v1.1.4

This package provides a convenient ArabTeX-like user-interface for typesetting
languages using the Arabic script in XeLaTeX, with flexible access to font
features. Input in ArabTeX notation can be set in three different
vocalization modes or in roman transliteration. Direct UTF-8 input is also
supported.  Since most of the ‘real work’ -- namely parsing and converting
ArabTeX input to Unicode -- is done at the level of TECkit mappings,
ArabXeTeX is really performant.

The TECkit fontmapping files (*.map and *.tec) should be copied to 
<TEXMF>/fonts/misc/xetex/fontmapping/arabxetex/

-------------------------
François Charette
© 2007–2010
%</readme>
%
% \fi
%
% \errorcontextlines=999
% \makeatletter
%
% \GetFileInfo{\jobname.sty}
% 
% \title{~\hfil {\color{IslamicGreen}\arabxetex \hfil \arabxetexAr}\hfil~\\ 
% An \arabtex-like interface for typesetting languages in Arabic script with \XeLaTeX}
% \author{François Charette}
% \date{\filedate \qquad \fileversion}
% 
% \maketitle
% \tableofcontents
% 
% 
% \DeleteShortVerb{\|}
% \MakeShortVerb{\¦}
% 
% \begin{abstract}
% This package provides a convenient \arabtex-like user-interface for typesetting 
% languages using the Arabic script in \XeLaTeX, with flexible access to font
% features. Input in \arabtex\ notation can be set in three different vocalization 
% modes or in roman transliteration. Direct UTF-8 input is also supported. 
% Since most of the ‘real work’ -- namely parsing and converting \arabtex\ input to Unicode -- 
% is done at the level of TECkit mappings, \arabxetex\ is really performant.
% \end{abstract}
% 
% \tableofcontents
% 
% \section{Introduction}
% 
% Since the early 1990s, \TeX\ users who wished to typeset in Arabic script 
% have relied on Klaus Lagally's excellent \arabtex\ system.\footnote{ ^^A
% 	\url{ftp://ftp.informatik.uni-stuttgart.de/pub/arabtex/arabtex.htm}.
% 	Version 2.00 was released in April 1992 and version 3.00 in November of
% 	the following year.  The latest stable version (dated 2 July 2006) is
% 	3.11s. Since version 3.02 the Hebrew language is also supported.}
% While \arabtex's overall qualities and Lagally's masterful \TeX-programming
% abilities are undeniable, the system can hardly hide its 15-odd years of
% existence.  Or to state it otherwise, it is now slowly becoming an archaic
% piece of sofware (which as a historian who has used \arabtex\ for more than
% eleven years I find a sad thing to admit). First of all, it is monolithic and
% idiosyncratic, in the sense that everything -- from parsing the input encoding,
% to doing contextual analysis, to assembling all elements of the script and
% placing them on the page from right to left, in defiance of \TeX's natural
% logic -- is taken care of by means of \TeX\ macro programming.  Thus before the
% availability of relatively fast Pentium processors, \arabtex\ was really slow,
% but this is less of an issue nowadays. Another disadvantage is that one is
% inexorably bound to use the custom Naskh font, which without being necessarily
% ugly does not meet the standards of fine typesetting. Finally, the collection
% of very sophisticated \TeX\ code that constitutes the \arabtex\ package is not
% documented at all, which means that even though it is now licensed under the
% LPPL, its internals are extremely difficult to understand in depth.  
% (It is only fair to state, however, that Prof.~Lagally has been extremely 
% responsive to ideas, suggestions and wishes by users in the past).
% 
% The introduction of Omega in \ca 1996, a \TeX\ extension for multilingual,
% multidirectional typesetting based on Unicode, raised of course many hopes, but
% these have by now completely dissipated since the project has been moribund for
% several years and is now probably defunct. Aleph, a more humble attempt to reach 
% some of the goals set by Omega, has been equally unsuccessful.\footnote{ ^^A
% 	Based on Omega 1.15 and \eTeX, Aleph attempted to provide a usable and
% 	stable branch.  See
% 	\url{http://www.tug.org/TUGboat/Articles/tb25-1/bilotta.pdf}.
% 	Unfortunately the project is currently dormant.}
% Both engines suffered of numerous bugs and never reached a stable and usable
% state, \textit{at least according to my own subjective experience}.
% 
% After completion of the initial alpha version of \arabxetex, I came across the
% package \pkg{Arabi} by Youssef Jabri on CTAN.\footnote{ ^^A
% 	\url{http://www.ctan.org/tex-archive/language/arabic/arabi/}}
% This is a pure LaTeX/Babel approach to Arabic typesetting which makes use of
% TFM hackery (by means of complex sequences of ligature rules) to provide custom
% contextual analysis for individual fonts. Thus the \pkg{Arabi} only works in
% combination with the fonts for which support is built in the package.
% 
% With the availability of Jonathan Kew's \XeTeX, users can now rely on a very
% up-to-date typesetting engine based on the integration of \eTeX, Unicode, and 
% modern font-rendering technology (AAT and ICU), without the complex hassle of
% font installation.\footnote{ ^^A
% 	\url{http://scripts.sil.org/xetex}. \XeTeX\ was originally developed 
% 	for the Mac~OS~X platform, but Linux and WIN32 ports are now available.}
% 
% \section{Description}
% 
% The \arabxetex\ package consists of a set of \textbf{TECkit}\footnote{ ^^A
% 	\url{http://scripts.sil.org/TECkit}} 
% mappings for converting internally from \arabtex's \ASCII\ input convention to
% Unicode, and a \LaTeX\ style file (\pkg{arabxetex.sty}) that provides a
% convenient user interface for typesetting in those languages. 
% For details on \arabtex\ and its input convention the reader is referred to 
% Lagally's detailed manual.\footnote{ ^^A
% 	\url{http://129.69.218.213/arabtex/doc/arabdoc.pdf}}
% \arabxetex\ introduces several additions, and a few minor modifications, 
% to \arabtex's conventions. These are documented in tabular form in 
% Section~\ref{conventiontable}.  \arabxetex\ relies on the package \pkg{bidi}
% which contains the macros necessary for bidirectional typesetting --
% using \eTeX's \cmd{\beginR} \ldots \cmd{\endR}, \cmd{\beginL} \ldots \cmd{\endL} 
% primitives.  The main code in \pkg{bidi.sty} is essentially borrowed, 
% with minor adaptations, from the file \pkg{rlbabel.def} in the Babel
% distribution (where it serves for typesetting Hebrew). Some improvements 
% in the beta version were inspired by Youssef Jabri's \pkg{Arabi}.
% 
% Languages supported at present are the same as in \arabtex, namely: Arabic,
% Maghribi Arabic, Farsi (Persian), Urdu, Sindhi, Kashmiri, Ottoman Turkish,
% Kurdish, Jawi (Malay) and Uighur.  \arabxetex\ adds support for several
% additional Unicode characters, so that some more languages are probably
% supported de-facto as well (such as Western Punjabi).
% 
% \subsection{The \arabtex\ input encoding}
% 
% Why would one need to type those languages by means of an old-fashioned 
% \ASCII\ \label{arabtexnotes} representation?  
% Native speakers have access to keyboards that allow to input
% them directly in Unicode, while non-native scholars who need to type them 
% can use keymaps or keyboard utilities to achieve the same, so why bother? 
% An expert in Arabic mathematical linguistic and author of the Perl module 
% \pkg{Encode::Arabic::ArabTeX}, Otakar Srmž, has this to say about the 
% virtues of \arabtex\ \ASCII\ encoding:
% \begin{quote}
% 	“ArabTeX is an excellent extension to TeX/LaTeX designed for
% 	typesetting the right-to-left scripts of the Orient. It comes up with
% 	very intuitive and comprehensible lower \ASCII\ transliterations, the
% 	expressive power of which is even better than that of the scripts.”^^A\footnote{ %
% 		[Source: \url{http://search.cpan.org/~smrz/Encode-Arabic-1.16/ArabTeX/ArabTeX.pm#DESCRIPTION}.
% 		See also \url{http://ufal.mff.cuni.cz/~smrz/ICFP2006/icfp-encode.pdf}]^^A}
% \end{quote}
% There are indeed several advantages in using \arabtex\ input convention for
% typesetting in the Arabic script, besides ease and legibility.  First it is
% possible and even trivial to switch between a representation of the data in the
% Arabic script and its romanized transliteration, without having to modify it.
% Second, despite the increasing availability of good Unicode editors that
% support bidirectionality, editing (La)\TeX\ source files with bidirectional
% content can be a real pain, for it leads to confusion and distraction.  Also,
% with complex multi-layer documents such as critical editions, where footnotes
% and annotations abound, the advantage of dealing with a plain \ASCII\ encoding
% cannot be overstated. Finally, such an input convention allows far greater
% control of typographical details.
% 
% \section{Usage}
% 
% For short insertions (say within a LR text), use ¦\text<language>[option]{...}¦
% \DescribeMacro{\textarab}\DescribeMacro{\textfarsi}\marginpar{\raggedleft\small etc.}
% where ¦<language>¦ is one of the following (alternative names are given in brackets):
% \begin{compactitem}[\textbf{·}]
% 	\item ¦arab¦ (¦Arabic¦),\footnote{ ^^A
% 		Since the command \cmd\arabic\ is already defined in \LaTeX, we chose
% 		the environment ¦arab¦ and the command \cmd\textarab\ instead,
% 		but the equivalent environment ¦Arabic¦ and the command
% 		\cmd\textarabic\ are also available.}
% 	\item ¦farsi¦ (¦persian¦), 
% 	\item ¦urdu¦, 
% 	\item ¦sindhi¦, 
% 	\item ¦pashto¦, 
% 	\item ¦ottoman¦ (¦turk¦),
% 	\item ¦kurdish¦,
% 	\item ¦kashmiri¦, 
% 	\item ¦malay¦ (¦jawi¦), and 
% 	\item ¦uighur¦.
% \end{compactitem}
% For typesetting whole paragraphs in Arabic script, use 
% \DescribeMacro{\begin\{arab\}}\DescribeMacro{\begin\{farsi\}}\marginpar{\raggedleft\small etc.}
% \begin{Verbatim}[gobble=2,fontsize=\normalsize]
%   \begin{<language>}[option] 
%   ...
%   \end{<language>}
% \end{Verbatim}
% 
% Most languages can be typeset in one of four modes: non-vocalized, vocalized,
% fully-vocalized, and transliterated, exactly as in \arabtex, and these are
% switched by means of the options ¦novoc¦, ¦voc¦, ¦fullvoc¦ and ¦trans¦,
% respectively.  Note that Kurdish and Uighur only have one vocalization mode.
% The mode can be determined either globally as an option to the \pkg{arabxetex}
% package, or as a local option of either the ¦\text<language>¦ commands or
% ¦\begin{<language>}¦ \ldots ¦\end{<language>}¦ environments. 
% When no option is set locally, the global option is chosen.
% The default global option is always \textit{non-vocalized} mode.
% 
% There is also an option \pkg{utf} for input in plain UTF-8 encoding.  Note that
% it is also possible to mix \arabtex\ input convention and UTF-8 characters,
% since the latter will not be affected by the font-mapping (except of course in
% transliteration mode, unless your roman font also contains Arabic characters).
% There are also advantages of choosing an \arabtex\ mapping (\ie one of the
% ¦novoc¦, ¦voc¦ and ¦fullvoc¦ modes) even with UTF-8 encoding, as it provides
% easy access to special glyphs and some useful features. See further below.
% 
% Left-to-right insertions in Latin script within an Arabic environment 
% can be made by means of the \DescribeMacro{\textLR} 
% command \cmd\textLR¦{…}¦.\new{1.1.2}\footnote{ %
%	In version 1.0 this was called \cmd\textlatin, which was renamed \cmd\textroman
%	in version 1.1. But the former conflicted with \pkg{Babel}, 
%	and the latter with \pkg{Beamer}. Hopefully \cmd\textLR\ 
%	won't be as short-lived!}
% Emphasis in Arabic is traditionally indicated by
% overlining the text, and this can be achieved with the command
% \DescribeMacro{\aemph} 
% \cmd\aemph:\footnote{ ^^A
% 	This macro makes use of the mathematical command \cmd\overline,
% 	which explains why \pkg{arabxetex} requires the \pkg{amsmath} package.
% 	I welcome any suggestion that would make it possible to circumvent this dependency
% 	by emulating \cmd\overline\ directly.}
% 
% \parindentoff
% \begin{minipage}[t]{.4\textwidth}
% \begin{Verbatim}[gobble=2]
%     \begin{arab}[novoc]
%     mi_tAl: \aemph{45} darajaT	
%     \end{arab}
% \end{Verbatim}
% \end{minipage}
% \hfill
% \begin{minipage}[t]{.4\textwidth}
% \begin{arab}[novoc]
% 	~ \\
% mi_tAl: \aemph{45} darajaT	
% \end{arab}
% \end{minipage}
% \medskip
%
%\subsection{Font setup}
%
% \arabxetex\ requires the user to define an \cmd\arabicfont\ in the preamble.  
% \DescribeMacro{\arabicfont}
% The recommended procedure, by means of \pkg{fontspec}, is to define it as follows:
% \begin{Verbatim}[gobble=2,fontsize=\normalsize]
%     \newfontfamily\arabicfont[Script=Arabic]{<fontname>}
% \end{Verbatim}
% If not, then a warning is issued and \arabxetex\ will attempt to load 
% the font \textbf{Scheherazade} (designed by Bob Halissy and
% Jonathan Kew of SIL International).\footnote{ 
% 	Available from \url{http://scripts.sil.org/ArabicFonts}. For
% 	typesetting Sindhi and Kashmiri, the font \textbf{Lateef}, available at
% 	the same place, is also recommended.}
% You can also define specific fonts \new{1.1} for all other languages, by similarly defining
% ¦\<language>font¦, such as for example:
% \begin{Verbatim}[gobble=2,fontsize=\normalsize]
%     \newfontfamily\urdufont[Script=Arabic]{Nafees Pakistani Naskh}
% \end{Verbatim}
% So for example if the \cmd\texturdu\ command or the ¦urdu¦ environment is used,
% \arabxetex\ will use the \cmd\urdufont\ if it is defined, 
% and the \cmd\arabicfont\ if not. In the same manner one can define 
% \cmd\maghribifont, \cmd\ottomanfont, \cmd\uighurfont, etc.
% 
% 
% 
% 
% \subsection{Examples}
% 
% \subsubsection{Contextual analysis of \textit{hamza}}
%
% As with \arabtex, a contextual analysis of the input encoding is performed
% (at the font-mapping level) to automatically determine the carrier of the 
% \textit{hamza}, as illustrated by the following examples:
% 
% \begin{Verbatim}[gobble=2,fontsize=\normalsize]
% \begin{arab}
% 'amruN, 'ibiluN, 'u_htuN, '"u_ht"uN, '"Uql"Id"Is, ra'suN, 'ar'asu, 
% sa'ala, qara'a, bu'suN, 'ab'usuN, ra'ufa, ru'asA'u, bi'ruN, 'as'ilaTuN, 
% ka'iba, qA'imuN, ri'AsaTuN, su'ila, samA'uN, barI'uN, sU'uN, bad'uN, 
% ^say'uN, ^say'iN, ^say'aN, sA'ala, mas'alaTuN, saw'aTuN, _ha.tI'aTuN, 
% jA'a, ridA'uN, ridA'aN, jI'a, radI'iN, sU'uN, .daw'uN, qay'iN, .zim'aN
% , yatasA'alUna, 'a`dA'akum, 'a`dA'ikum, 'a`dA'ukum maqrU'aT, mU'ibAt, 
% taw'am, yas'alu, 'a.sdiq^A$\;$'uh_u, ya^g^I'u, s^U'ila 
% \end{arab}
% \end{Verbatim}
% 
% \begin{arab}[voc]
% 'amruN, 'ibiluN, 'u_htuN, '"u_ht"uN, '"Uql"Id"Is, ra'suN, 'ar'asu, 
% sa'ala, qara'a, bu'suN, 'ab'usuN, ra'ufa, ru'asA'u, bi'ruN, 'as'ilaTuN, 
% ka'iba, qA'imuN, ri'AsaTuN, su'ila, samA'uN, barI'uN, sU'uN, bad'uN, 
% ^say'uN, ^say'iN, ^say'aN, sA'ala, mas'alaTuN, saw'aTuN, _ha.tI'aTuN, 
% jA'a, ridA'uN, ridA'aN, jI'a, radI'iN, sU'uN, .daw'uN, qay'iN, .zim'aN
% , yatasA'alUna, 'a`dA'akum, 'a`dA'ikum, 'a`dA'ukum maqrU'aT, mU'ibAt, 
% taw'am, yas'alu, 'a.sdiq^A$\;$'uh_u, ya^g^I'u, s^U'ila 
% \end{arab}
% 
% \subsubsection{Special orthographies}
% 
% Some Arabic words, like \textit{miʾa} “hundred”, have irregular orthographies.
% \arabxetex\ recognizes them automatically.
% 
% \begin{minipage}[t]{.4\textwidth}
% \begin{Verbatim}[gobble=2]
%     \begin{arab}[voc]
%     mi'aT , mi'at"An , sab`ami'"aT
%     \end{arab}
% \end{Verbatim}
% \end{minipage}
% \hfill
% \begin{minipage}[t]{.4\textwidth}
% \begin{arab}[voc]
% ~ \\
% mi'aT , mi'at"An , sab`ami'"aT
% \end{arab}
% \end{minipage}
% \bigskip
% 
% \textbf{NB}: For the time being only \textit{miʾa} is supported, but more irregular
% constructs should be added in later versions.
% 
% \subsubsection{Typesetting the Holy Qurʾān}
%
% High-quality typesetting of the Holy Qurʾān (\textarab{al-qur'An al-karIm}) is a 
% most complex and demanding task, which probably should be left to professional 
% typesetters. Nevertheless, with Open Type fonts that cover the full Unicode range 
% for the Arabic script, it is possible to achieve pretty decent results.
% The following examples represent my attempt to reproduce, with the font
% Scheherazade, the various typographic features of a typical printed edition 
% from Saudi Arabia. 
% 
% In printed Qurʾāns, one frequently encounters typographical oddities that are 
% not taken into account by Unicode, such as the \textit{hamza} placed directly over 
% the baseline instead as over the \textit{alif}. 
% But with a \TeX\ macro it is possible to emulate this rather well: 
% 
% \begin{Verbatim}[gobble=2,fontsize=\small]
% \newcommand{\hamzaB}{\char"200D\char"0640\raise-.95ex\hbox{\char"0654}\char"200D}
%
% \begin{arab}[fullvoc]
% mina 'l-qur'Ani 'l-karImi, sUraTu 'l-ssajdaTi 15--16:
%  
% 'innamA yu'minu bi-\hamzaB a|"Ay___atinA 'lla_dIna 'i_dA _dukkirUA bihA 
% _harrUA sujjadaN wa-sabba.hUA bi-.hamdi rabbihim wa-hum lA yastakbirUna 
% SAJDA [[15]] tatajAfY_a junUbuhum `ani 'l-ma.dAji`i yad`Una rabbahum 
% _hawfaN wa-.tama`aN wa-mimmA razaqn_ahum yunfiqUna [[16]]
% \end{arab}
%
% \begin{arab}[fullvoc]
% sUraTu 'l-baqaraTi 71--72:
% 
% qAla 'innahu, yaqUlu 'innahA baqaraTuN llA _dalUluN tu_tIru 'l-'ar.da wa-lA
% tasq.I 'l-.har_ta musallamaTuN llA ^siyaTa fIhA|^JIM qAluW" 'l-\hamzaB a___ana
% ji'ta bi-'l-.haqqi|^JIM fa_daba.hUhA wa-mA kAdduW" yaf`alUna [[71]] wa-'i_d
% qataltum nafsaN fa-udda$\,$_ara|'|_i"tum fIhA|^SLY wa-al-ll_ahu mu_hrijuN mmA
% kun"tum taktumUna [[72]] 
% \end{arab} 
% \end{Verbatim}
% 
% \begin{arab}[fullvoc]
% mina 'l-qur'Ani 'l-karImi, sUraTu 'l-ssajdaTi 15--16:
%  
% 'innamA yu'minu bi-\hamzaB a|"Ay___atinA 'lla_dIna 'i_dA _dukkirUA bihA _harrUA
% sujjadaN wa-sabba.hUA bi-.hamdi rabbihim wa-hum lA yastakbirUna SAJDA [[15]]
% tatajAfY_a junUbuhum `ani 'l-ma.dAji`i yad`Una rabbahum _hawfaN wa-.tama`aN
% wa-mimmA razaqn_ahum yunfiqUna [[16]]
% \end{arab}
%
% \begin{arab}[fullvoc]
% sUraTu 'l-baqaraTi 71--72:
% 
% qAla 'innahu, yaqUlu 'innahA baqaraTuN llA _dalUluN tu_tIru 'l-'ar.da wa-lA
% tasq.I 'l-.har_ta musallamaTuN llA ^siyaTa fIhA|^JIM qAluW" 'l-\hamzaB a___ana
% ji'ta bi-'l-.haqqi|^JIM fa_daba.hUhA wa-mA kAdduW" yaf`alUna [[71]] wa-'i_d
% qataltum nafsaN fa-udda$\,$_ara|'|_i"tum fIhA|^SLY wa-al-ll_ahu mu_hrijuN mmA
% kun"tum taktumUna [[72]] 
% \end{arab}
% 
% \subsubsection{Farsi}
%
% These are some of the Persian examples in the \arabtex\ documentation,
% typeset with the font Simple Farsi Bold:
% \begin{Verbatim}[gobble=2,fontsize=\normalsize]
% \begin{farsi}[voc]
% _hwAb, xwI^s, _hwod, ^ceH, naH, yal_aH, _hAneH, _hAneHhA, _hAneH-hA, 
% ketAb-e, U, rAh-e,   t_U, nAmeH-i, man, bInI-e, An, mard, pA-i, In, 
% zan, bAzU-i, In, zan, dAr-_i, man, _hU-_i, t_U, nAmeH-_i, sormeH-_i, 
% gofteH-_i, ketAb-I, rAh-I, nAmeH-I, dAnA-I, pArU-I, dAnA-I-keH, 
% pArU-I-keH, rafteH-am, rafteH-Im, AnjA-st, U-st, t_U-st, ketAb-I-st, 
% be-man, be-t_U, be-An, be-In, be-insAn, beU, be-U, .sA.heb"|_hAneH, 
% pas"|andAz, naw"|AmUz  
% \end{farsi} 
% \end{Verbatim}
% 
% \begin{farsi}[voc]
% _hwAb, xwI^s, _hwod, ^ceH, naH, yal_aH, _hAneH, _hAneHhA, _hAneH-hA, 
% ketAb-e, U, rAh-e,   t_U, nAmeH-i, man, bInI-e, An, mard, pA-i, In, 
% zan, bAzU-i, In, zan, dAr-_i, man, _hU-_i, t_U, nAmeH-_i, sormeH-_i, 
% gofteH-_i, ketAb-I, rAh-I, nAmeH-I, dAnA-I, pArU-I, dAnA-I-keH, 
% pArU-I-keH, rafteH-am, rafteH-Im, AnjA-st, U-st, t_U-st, ketAb-I-st, 
% be-man, be-t_U, be-An, be-In, be-insAn, beU, be-U, .sA.heb"|_hAneH, 
% pas"|andAz, naw"|AmUz  
% \end{farsi}
% 
% \subsubsection{Urdu}
% 
% An Urdu example, typeset with Nafees Pakistani Naskh:\footnote{ ^^A
% 	The example is borrowed from 
% 	\url{http://tabish.freeshell.org/u-trans/urducode.html}}
% \begin{Verbatim}[gobble=2,fontsize=\normalsize]
% \begin{urdu}[novoc]
% ,ham `i^sq kE mArO.n kA itnA ,hI fasAna,h ,hae
% rOnE kO na,hI.n kO'I ,ha.nsnE kO zamAna,h ,hae
% 
% ya,h kiskA ta.sawwur ,hae ya,h kiskA fasAna,h ,hae
% jO a^sk ,hae A.nkhO.n mE.n tasbI.h kA dAnA ,hae
% \end{urdu}
% \end{Verbatim}
% 
% \begin{urdu}[novoc]
% ,ham `i^sq kE mArO.n kA itnA ,hI fasAna,h ,hae\\
% rOnE kO na,hI.n kO'I ,ha.nsnE kO zamAna,h ,hae
% 
% ya,h kiskA ta.sawwur ,hae ya,h kiskA fasAna,h ,hae\\
% jO a^sk ,hae A.nkhO.n mE.n tasbI.h kA dAnA ,hae
% \end{urdu}
% ^^Aﮨﹷﻢ ﻋﹻﺸﻖ ﻛﮯ ﻣﺎﺭﻭﮞ ﰷ ﹺﺍﺗﻨﺎ ﮨﹻﯽ ﻓﹷﺴﺎﻧﹷﻪ ﮨﹷﮯ
% ^^Aﺭﻭﻧﮯ ﻛﻮ ﻧﹷﮩﹻﻴﮟ ﻛﻮﺋﹻﯽ ﮨﹷﻨﺴﻨﮯ ﻛﻮ ﺯﹶﻣﺎﻧﹷﻪ ﮨﹷﮯ
% 
% ^^Aﻳﹷﻪ ﻛﹻﺴﲀ ﺗﹷﺼﹷﻮﹽﹸﺭ ﮨﹷﮯ ﻳﹷﻪ ﻛﹻﺴﲀ ﻓﹷﺴﺎﻧﹷﻪ ﮨﹷﮯ
% ^^Aﺟﻮ ﺍﺷﻚ ﮨﹷﮯ ﺁﻧﻜﮭﻮﮞ ﻣﻴﮟ ﺗﹷﺴﺒﹻﻴﺢ ﰷ ﺩﺍﻧﺎ ﮨﹷﮯ
% \renewcommand{\urdufont}{\arabicfont}	
% \renewcommand{\farsifont}{\arabicfont}
%
% \subsection{Special considerations}
% \parindenton
% \subsubsection{The name of God}  \label{allahliga}
% 
% The glyph \textsf{FDF2}, defined as ‘\textsc{arabic ligature allah isolated form}’ 
% by the Unicode Consortium, is a source of great confusion.  
% It is displayed in the Unicode Book with an initial alif and thus represents the 
% name of God, Allāh, which in Arabic is always written as a special ligature 
% (\ie {\tradarabic اﷲ} and not {\arabicfont الله}).\footnote{ ^^A
% 	The glyph {\tradarabic اﷲ} is taken from the font Traditional Arabic.
% 	In Scheherazade its design is rather suboptimal: {\arabicfont\char"FDF2}.}
% 
% However, a substantial portion of real-world fonts rather represent that 
% ligature \textit{without} the initial alif.\footnote{ ^^A
% 	My research on Arabic fonts available or known to me yields the following picture: 
% 	The fonts that do not display the initial alif in the ligature \textsf{FDF2} 
% 	include those provided by \href{http://www.linotype.com/2517/arabicfonts.html}{Linotype}; 
% 	the great majority of those licensed to or developed by 
% 	\href{http://www.microsoft.com/typography/links/FontPortal.aspx?PID=8}{Microsoft}
% 	(I could verify it for Times New Roman, Arial, Courier New, Microsoft Sans Serif, 
% 	Arabic Transparent, Simplified Arabic, Simplified Arabic Fixed, 
% 	WinSoft Serif Pro, Traditional Arabic, Andalus, Old Antic Bold, 
% 	Old Antic Decorated and Farsi Simple Bold); the fonts distributed by 
% 	\href{http://www.arabeyes.org}{Arabeyes.org}; 
% 	\href{http://scripts.sil.org/ArabicFonts}{SIL}'s Lateef; 
% 	and the fonts developed by \href{www.crulp.org}{CRULP} in Pakistan.
% 	The Unicode-conformant fonts, on the other hand, are:
% 	\href{http://scripts.sil.org/ArabicFonts}{SIL}'s Scheherazade, 
% 	\href{http://www.tdc.org/news/2006Results/AdobeArabic.html}{Adobe Arabic} 
% 	(distributed with the \href{http://www.adobe.com/ceea/}{Middle-Eastern version} 
% 	of the latest \href{http://www.adobe.com/products/acrobat/readermain.html}{Adobe Reader 7}),
% 	Arial Unicode MS, and \href{http://sakkal.com/type/typesetting.html}{Arabic Typesetting}
% 	(distributed with \href{http://www.microsoft.com/typography/VOLT.mspx}{VOLT} 
% 	and with \href{http://www.microsoft.com/office/editions/prodinfo/language/proofingtools.mspx}{Microsoft Office Proofing Tools 2003}).}
% The confusion probably has to do with Unicode’s omission to include the ALLAH
% ligature without the initial alif, which is imperatively required for typesetting 
% expressions like \textit{al-ḥamdu li-llāh} \RL{\tradarabic الحمد ﷲ}. 
% Many fonts code the ligature {\tradarabic ﷲ} in the Private Use Area
% and make it accessible as a default ligature for the sequence \textit{lām-lām-hāʾ}. 
% In such a case it can be generated from the input ¦\textarab{l|lh}¦ 
% (the vertical bar here tells \arabxetex\ not to interpret the sequence ¦ll¦ 
% as \textit{lām} with \textit{shadda}).\footnote{ ^^A
% 	This currently does not work with Scheharazade, but the 
% 	developers are aware of the issue.}
% Besides not being a standard \arabtex\ input sequence, the trouble is that 
% there is no way to know a priori whether the font provides the glyph {\tradarabic ﷲ}
% at all, and whether it is defined as a default ligature from the above input.
% 
% To solve this problem with \arabxetex, we had no choice but provide two classes of 
% font mappings, one for each of the above two categories of Arabic fonts.\footnote{ ^^A
% 	This has the unfortunate consequence of doubling the number of mappings shipped
% 	with \arabxetex. But since their size is small, it is more an esthetic annoyance 
% 	than a practical one.}
% By default we assume the canonical situation where \textsf{U+FDF2} corresponds to 
% {\tradarabic اﷲ}, but the user can change this by choosing one of the package options
% ¦fdf2alif¦ or ¦fdf2noalif¦; it can also be changed locally by means of the commands
% \DescribeMacro{\SetAllahWithAlif}
% \DescribeMacro{\SetAllahWithoutAlif}
% \cmd\SetAllahWithAlif\ and \cmd\SetAllahWithoutAlif\ \textit{before} making use of 
% \cmd\arabicfont. In a future version we might perhaps implement a database of Arabic fonts 
% within \arabxetex, so that this would work automatically in the most common cases.
% 
% The font Adobe Arabic has in addition the ligature \textit{fa-li-llāh}: 
% \textit{fā fatḥa lām kasra lām hāʾ} $\rightarrow$ {\adobearabic فَلِله}, which can be
% entered in \arabtex\ notation as ¦\textarab{falilh}¦. In this particular case we also
% provide the input convention ¦FALILLAH¦.
% 
% 
% ^^ATEST Scheherazade: \textarab{l|lh} 
% ^^A
% ^^A\renewcommand{\arabicfont}{\tradarabic} 
% ^^ATEST Trad Arabic: \textarab{l|lh}
% 
% ^^ALinotype: \\
% ^^ALotus Linotype {\lotusfont\char"FDF2} --- 
% ^^AAlHarfAlJadid Linotype One {\fontspec[Scale=1.5]{AlHarfAlJadid Linotype One}\char"FDF2} --- 
% ^^AAlHarfAlJadid Linotype Two {\fontspec[Scale=1.5]{AlHarfAlJadid Linotype Two}\char"FDF2} --- 
% ^^AMariam Linotype {\fontspec[Scale=1.5]{Mariam Linotype}\char"FDF2} --- 
% ^^AQadi Linotype {\fontspec[Scale=1.5]{Qadi Linotype}\char"FDF2}  
% ^^A
% ^^AFonts from or licensed to Microsoft: \\
% ^^ATimes New Roman {\fontspec[Scale=1.5]{Times New Roman}\char"FDF2} --- 
% ^^AArial {\fontspec[Scale=1.5]{Arial}\char"FDF2} --- 
% ^^ACourier New {\fontspec[Scale=1.5]{Courier New}\char"FDF2} --- 
% ^^AMicrosoft Sans Serif {\fontspec[Scale=1.5]{Microsoft Sans Serif}\char"FDF2} --- 
% ^^AArial Unicode MS {\arialuni\char"FDF2} --- 
% ^^AArabic Transparent {\fontspec[Scale=1.5]{Arabic Transparent}\char"FDF2} --- 
% ^^ASimplified Arabic {\fontspec[Scale=1.5]{Simplified Arabic}\char"FDF2} --- 
% ^^ASimplified Arabic Fixed {\fontspec[Scale=1.5]{Simplified Arabic Fixed}\char"FDF2} --- 
% ^^AWinSoft Serif Pro {\fontspec[Scale=1.5]{WinSoft Serif Pro}\char"FDF2} --- 
% ^^ATraditional Arabic {\fontspec[Scale=1.5]{Traditional Arabic}\char"FDF2} --- 
% ^^AArabic Typesetting {\arabtype\char"FDF2} --- 
% ^^AAndalus {\fontspec[Scale=1.5]{Andalus}\char"FDF2} --- 
% ^^AOld Antic Bold {\fontspec[Scale=1.5]{Old Antic Bold}\char"FDF2} --- 
% ^^AFarsi Simple Bold {\fontspec[Scale=1.5]{Farsi Simple Bold}\char"FDF2}  
% ^^A
% ^^A
% ^^AAdobe:\\
% ^^AAdobe Arabic {\adobearabic\char"FDF2} --- 
% ^^A
% ^^ASIL: \\
% ^^AScheherazade {\arabicfont\char"FDF2} --- 
% ^^ALateef {\fontspec{Lateef}\char"FDF2}  
% ^^A
% ^^A%CRULP:\\
% ^^A%Nafees Nastaleeq {\fontspec{Nafees Nastaleeq}\char"FDF2} --- 
% ^^A%{\fontspec{Nafees Pakistani Naskh}\char"FDF2}
% ^^A
% ^^AArabeyes:\\
% ^^AKacstBook {\fontspec{KacstBook}\char"FDF2}
% ^^AKacstFarsi {\fontspec{KacstFarsi}\char"FDF2}
% 
% 
% 
% 
% \subsection{Transliteration}
% 
% At the moment transliteration mappings are provided for Arabic, Persian, Urdu, 
% Sindhi and Pashto. The rest may be provided in a future version.
% As a rule the default conventions provided are those of the Library of Congress.
% \new{1.1.2}For Arabic the alternative transliteration of the 
% \href{http://www.orientasia.info/download/arab_trans.pdf}{Deutsche Morgenländische Gesellschaft}
% is also available (but should be still considered experimental). 
% You can set it with the command \DescribeMacro{\SetTranslitConvention}
% ¦\SetTranslitConvention{dmg}¦.
% To switch back to the Library of Congress transliteration, type
% ¦\SetTranslitConvention{loc}¦.
% Additional conventions for other languages, as with \arabtex, \eg 
% Encyclopedia of Islam, Encyclopedia Iranica, etc., might be added later.\footnote{ %
%     It is suggested that you contact the author if you have such needs.}
%
% Transliteration is set in italics by default. This can be changed by declaring, e.g.,
% \DescribeMacro{\SetTranslitStyle}
% ¦\SetTranslitStyle{\upshape}¦.
% To transliterate proper nouns with capitals, prefix the words to be capitalized with \cmd\UC:
% \DescribeMacro{\UC}
% \begin{Verbatim}[gobble=2,fontsize=\normalsize]
% \begin{arab}[trans]
% al-^say_h al-`Alim \UC na.sIr \UC al-dIn \UC al-.tUsI
% \end{arab}
% \end{Verbatim}
% ^^A\SetTranslitStyle{\gentium\itshape}
% \begin{arab}[trans]
% \noindent al-^say_h al-`Alim \UC na.sIr \UC al-dIn \UC al-.tUsI
% \end{arab}
% Note that the article \textit{al-} is automatically skipped.
% Note also that since the transliteration is coded in Unicode at the level of
% the font-mapping, it is necessary that the font contains all required glyphs.\footnote{ ^^A
% 	See Appendix~\ref{latinextfonts} for a list of recommended Unicode fonts that cover the 
% 	full \textsc{latin extended additional} plane.}
% 
% ^^AIf your default roman font does not have the required
% ^^Adiacritics, you can declare for instance
% ^^A¦\SetTranslitStyle{\fontspec{Gentium}\itshape}¦.
% 
% ^^A\section{TODO}
% ^^A
% ^^A\begin{MYitemize}
% ^^A	\item implement romanized transliterations for kashmiri, turkish, kurdish, malay and uighur.
% ^^A	\item perhaps at a later stage: add support for hebrew and judeo-arabic?
% ^^A	\item syriac?
% ^^A\end{MYitemize}
% ^^A
% 
% \subsection{Typesetting critical editions with \pkg{ednotes}}
% \unsetfootnoteRL
% 
% In conjunction with \pkg{bidi}, the \pkg{ednotes} package makes it wonderfully
% easy to typeset critical editions of texts in Arabic script (or other RTL
% scripts).\footnote{ 
% 		The package \href{http://tug.ctan.org/tex-archive/macros/latex/contrib/ledmac/ledmac.pdf}{\pkg{ledmac}} 
%	has not been extensively tested with \pkg{bidi} yet, but our initial
%	trials were not successful.}
%	^^A Be aware that  -- as any package which plays with the primitive \cmd\everypar -- 
% 	^^A is not compatible with \pkg{bidi}.
% The direction of each level of footnotes can be controlled by 
% means of ¦\SetFootnoteHook{\setRL}¦ (or ¦\setLR¦) right before the declaration
% ¦\DeclareNewFootnote{X}[…]¦. Here is an example preamble that might be used
% for typesetting a critical edition with \pkg{ednotes} and \arabxetex.\footnote{ ^^A
% 	Of course if the edition is typed directly in UTF-8 encoding, 
% 	the use of \arabxetex\ is by no means compulsory. Yet see the 
% 	remarks in §~\ref{arabtexnotes} above.}
% See the \pkg{ednotes} documentation for more details.\footnote{ ^^A
% 	\url{http://www.ctan.org/tex-archive/macros/latex/contrib/ednotes/ednotugb.pdf}.
% 	See also \url{http://www.webdesign-bu.de/uwe_lueck/critedltx.html}.}
% 
% \begin{Verbatim}[gobble=2,fontsize=\small]
% \usepackage[modulo,perpage,para*]{ednotes}%this calls manyfoot.sty and lineno.sty 
% \usepackage{arabxetex}
% 
% % make \footnoterule of \textwidth
% \makeatletter%
% \renewcommand{\footnoterule}{\kern-3\p@
%   \hrule width \textwidth \kern 2.6\p@}
% \makeatother
% 
% \modulolinenumbers[5]
% % this is to set linenumbering in Arabic:
% \renewcommand{\linenumberfont}{\arabicfont\addfontfeature{Mapping=arabicdigits}\tiny}
% 
% \renewcommand{\extrafootnoterule}{}
% \SelectFootnoteRule[0]{extra}
% \SetFootnoteHook{\unsetRL}%--> must appear immediately before \DeclareNewFootnote 
% %% For ednotes the command \PrecedeLevelWith{X}{hook} is available
% \DeclareNewFootnote{B}[fnsymbol]
% \renewcommand*{\differentlines}[2]{\linesfmt{\RL{#1$-$#2}}}%
% \renewcommand*{\linesfmt}[1]{\raisebox{1ex}{\linenumberfont #1}~}%
% \renewcommand{\lemmafmt}[1]{#1~[ }%
% %Custom macros to enter variants, additions, omissions, illegible passages,
% % text above the line, marginal notes, lacunae, and restorations:
% \newcommand{\VAR}[2]{\Anote{\textarab{#1}}{\textarab{#2}}}
% \newcommand{\ADD}{\textroman{\textbf{+}}\,}% or \textarab{zAyid fI}\ 
% \newcommand{\OM}{\textroman{\textbf{\char"2013}}\,}% or \textarab{nAqi.s fI}\ 
% \newcommand{\ILLEG}{\textarab{.gayr maqrU' fI}\ }
% \newcommand{\BLANK}{\textarab{bayA.d fI}\ }
% \newcommand{\SUPERLIN}{\textarab{ta.ht al-sa.tr fI}\ }
% \newcommand{\MARG}{\textarab{bi-al-hAmi^s fI}\ }
% \newcommand{\LACUNA}{\textroman{\textlangle~{\dots}~\textrangle}}
% \newcommand{\RESTOR}[1]{\textroman{\textrangle}#1\textroman{\textlangle}}
% % to separate lemmas in different manuscripts:
% \def\SEP{\enskip$\Vert$\enskip} 
% \end{Verbatim}
% 
% For a real-life example of how to use ednotes with \arabxetex, see
% the file ¦ednotes_example.tex¦ which comes with this package.
% 
% 
% \section{Tabular overview of \arabtex\ encoding conventions} 
% \label{conventiontable}
% 
% The table is arranged alphabetically following the most signicant letter of the
% \ASCII\ input code.  Color convention: red means that the glyph is the default
% for the given input code, and that it is available in all languages except
% those where different glyphs are shown (in black). That default glyph is also
% displayed in light gray under each language in which it is featured.  Glyphs in
% blue are archaic forms (\eg old Urdu). An asterisk after the Unicode number
% means that the character was not available with \arabtex.  Green glyphs are
% special: either they are used to represent defective writing or they provide
% characters for other languages.  Those shown in the column for Arabic are
% available by default.  See the relevant notes at the end.
% 
% \begin{center}\footnotesize
% \renewcommand{\arraystretch}{2}
% ^^A                 1                 2      3        4       5        6          7         8         9       10 
% ^^A\tablefirsthead{%
% ^^Acode    &       arab        	 & farsi & urdu & pashto & sindhi & ottoman & kurdish & kashmiri & malay & uighur \\\hline}
% \tablehead{^^A
% \hline
% ^^A\tiny code    &\tiny       arab        	 &\tiny farsi &\tiny urdu &\tiny pashto &\tiny sindhi &\tiny ottoman &\tiny kurdish &\tiny kashmiri &\tiny malay &\tiny uighur \\\hline}
% ^^A next line: language codes after ISO-639-1
% code    &       ar        	 & fa & ur & ps & sd & tr & ku & ks & ms & ug \\\hline}
% \tabletail{\hline}
% \begin{supertabular}{|c|c|c|c|c|c|c|c|c|c|c|c|}
% ^^A\tablelasttail{\hline}
% ¦a¦	&	\GR{Ba}{064E}&&&&&&\GN[kurdish]{a/Ba}{}&&&\\  
% ¦A¦	&	\GR{BA}{0627}&&&&&&\GN[kurdish]{A/BA}{}&&&\\    
% ¦.a¦	&	&&&&&&&\GN[kashmiri]{B.a}{0654}&&\\
% ¦.A¦	&	&&&&&&&\GN[kashmiri]{B.A}{0672}&&\\ ^^A(kashmiri)  
% ¦_a¦	&	\GN{B_a}{0670}&&&&&&&&&\\  
% ¦_A¦	&	\GN{B_A}{}&&&&&&&&&\\  
% ¦:a¦	&	&&&&&&&&&\GN[uighur]{:a}{} \\
% ¦b¦	&	\GR{b}{0628}&\GG{b}&\GG{b}&\GG{b}&\GG{b}&\GG{b}&\GG{b}&\GG{b}&\GG{b}&\GG{b} \\
% ¦B¦	&	\GR{B}{0640}&&&&&&&&&\\
% ¦.b¦	&	\GV{.b}{066E}&&&&&&&&&\\
% ¦:b¦	&	&&&&\GR[sindhi]{:b}{067B}&&&&&\\ ^^A(sindhi)  
% ¦bh¦	&       &&&&\GN[sindhi]{bh}{0680}&&&&&\\
% ¦c¦	&	&&&&\GR[sindhi]{c}{0681}&\GN[turk]{c}{062C}&&&\GN[malay]{c}{0686}&\\
% ^^A¦.c¦	&	\GN{.c}{}&&&&&&&&&\\ 
% ¦,c¦	&	&&&\GR[pashto]{,c}{0685}&&\GN[turk]{,c}{0686}&\GN[kurdish]{,c}{0686}&&&\\
% ¦^c¦	&	\GG{^c}&\GR{^c}{0686}&\GG{^c}&\GG{^c}&\GG{^c}&\GG{^c}&\GG{^c}&\GG{^c}&\GG{^c}&\GG{^c} \\  
% ¦^ch¦	&	&&&&\GN[sindhi]{^ch}{0687}&&&&&\\
% ¦:c¦	&	&&&\GR[pashto]{:c}{0682*}&&&&&&\\
% ¦.^c¦	&	\GV{.^c}{06BF*}&&&&&&&&&\\
% ¦d¦	&	\GR{d}{062F}&\GG{d}&\GG{d}&\GG{d}&\GG{d}&\GG{d}&\GG{d}&\GG{d}&\GG{d}&\GG{d} \\
% ¦.d¦	&	\GR{.d}{0636}&\GG{.d}&\GG{.d}&\GG{.d}&\GG{.d}&\GG{.d}&\GG{.d}&\GG{.d}&\GG{.d}&\GG{.d} \\
% ¦,d¦	&       &&\GN[urdu]{,d}{0688}&\GN[pashto]{,d}{0689}&\GR[sindhi]{,d}{068A}&&&&&\\ ^^Aurdu
% ¦.,d¦	&	&&\GV{.,d}{068B*}$^a$&&&&&&&\\ ^^AWestern Punjabi
% ¦^d¦	&	\GV{^d}{06EE*}&&&&\GB[sindhi]{^d}{068E*}&&&&&\\ 
% ¦_d¦	&	\GR{_d}{0630}&\GG{_d}&\GG{_d}&\GG{_d}&\GG{_d}&\GG{_d}&\GG{_d}&\GG{_d}&\GG{_d}&\\ 
% ¦:d¦	&       &&&&\GR[sindhi]{:d}{068F}&&&&&\\ ^^A(sindhi)  
% ¦::d¦	&       &&\GB[urdu]{::d}{0690*}&&&&&&&\\
% ¦dh¦	&	&&&&\GN[sindhi]{dh}{068C}&&&&&\\
% ¦,dh¦	&	&&&&\GN[sindhi]{,dh}{068D}&&&&&\\
% ¦e¦	&	&\GR{Be}{}&\GG{Be}&\GN[pashto]{Be}{0659}&\GG{Be}&&\GN[kurdish]{e/Be}{}&\GN[kashmiri]{Be}{06D2+0658}&&\GN[uighur]{Be}{06D0} \\
% ¦E¦	&	&\GR{BE}{}&\GN[urdu]{BE}{06D2}&\GN[pashto]{BE}{06D0}&&&\GN[kurdish]{E/BE}{}&\GN[kashmiri]{BE}{06D2}&&\\ 
% ¦ee¦	&	&&&\GN[pashto]{Bee}{}&&&&&&\\
% ¦ae¦	&	\GN{Bae}{}&&\GN[urdu]{Bae}{}&\GN[pashto]{Bae}{}&\GN[sindhi]{Bae}{}&&&&&\\ 
% ¦Ee¦	&	&&&\GN[pashto]{BEe}{06CD}&&&&&&\\ 
% ^^A¦.e¦	&	\GN{B.e}{}&&&&&&&&&\\  
% ^^A¦^e¦	&	\GN{B^e}{}&&&&&&&&&\\ 
% ¦_e¦	&	&\GR{B_e}{}&&&&&&&&\\ 
% ¦`e¦    &       &&&&&&\GN[kurdish]{`e}{}&&&\\
% ¦'E¦	&	&&\GR[urdu]{B'E}{06D3}&&&&&&&\\ 
% ^^A¦:e¦	&	\GN{B:e}{}&&&&&&&&&\\ 
% ¦f¦	&	\GR{f}{0641}&\GG{f}&\GG{f}&\GG{f}&\GG{f}&\GG{f}&\GG{f}&\GG{f}&\GG{f}&\GG{f} \\ ^^A maghribi: \GN[maghribi]{f}{}  
% ¦.f¦	&	\GV{.f}{06A1}&&&&&&&&&\\ 
% ¦g¦	&	&\GR{g}{06AF}&\GG{g}&\GG{g}&\GG{g}&\GG{g}&\GG{g}&\GG{g}&\GN[malay]{g}{0762}&\GG{g} \\
% ¦G¦	&	&&&\GR[pashto]{G}{06AB}&&&&&&\\  
% ¦.g¦	&	\GR{.g}{063A}&\GG{.g}&\GG{.g}&\GG{.g}&\GG{.g}&\GG{.g}&\GG{.g}&\GG{.g}&\GG{.g}&\\ 
% ¦:g¦	&	&&&&\GR[sindhi]{:g}{06B3}&&&&&\\   
% ¦.:g¦	&	&&&&\GB[sindhi]{.:g}{06B4*}&&&&&\\   
% ¦,g¦	&	\GV{,g}{06AC}$^b$ &&&&&&&&&\\ 
% ¦^g¦	&	\GR{^g}{062C}&\GG{j}&\GG{j}&\GG{j}&\GG{j}&\GN[turk]{^g}{06A0}&\GN[kurdish]{^g}{063A}&\GG{j}&\GN[malay]{^g}{06A0}&\GN[uighur]{^g}{063A} \\
% ¦gh¦	&	&&&&\GN[sindhi]{gh}{}&&&&&\\
% ¦h¦	&	\GR{h}{0647}&\GG{h}&\GN[urdu]{h}{06BE}&\GG{h}&\GG{h}&\GG{h}&\GG{h}&\GG{h}&\GG{h}&\GG{h} \\ 
% ¦H¦	&	&\GN[farsi]{-H}{0647}&\GN[urdu]{-H}{06C3}&\GN[pashto]{-H}{0647}&&&&&&\\
% ¦.h¦	&	\GR{.h}{062D}&\GG{.h}&\GG{.h}&\GG{.h}&\GG{.h}&\GG{.h}&\GG{.h}&\GG{.h}&\GG{.h}&\\  
% ¦,h¦	&	&&\GR[urdu]{,h}{06C1}&&&&&\GN[kashmiri]{,h}{}&&\\ 
% ¦_h¦	&	\GR{_h}{062E}&\GG{x}&\GG{x}&\GG{x}&\GG{x}&\GG{x}&\GG{x}&\GG{x}&\GG{x}&\\ 
% ¦i¦	&	\GR{Bi}{0650}&&&&&&&&&\GN[uighur]{-i-}{066E} \\   
% ¦I¦	&	\GN{BI}{064A}&\GR[farsi]{BI}{06CC}&\GG[urdu]{BI}&\GG[pashto]{BI}&\GN[sindhi]{BI}{}&\GG[turk]{BI}&\GN[kurdish]{I/BI}{}&\GG[kashmiri]{BI}&\GN[malay]{BI}{}&\\
% ¦.I¦	&	\GR{B.I}{06CC*}&&&&&&&&&\\  
% ^^A¦.i¦	&	\GN{B.i}{}&&&&&&&&&\\     
% ¦_i¦	&	\GR{B_i}{0656}&&&&&&&&&\\ 
% ^^A¦:i¦	&	\GN{B:i}{}&&&&&&&&&\\  
% ¦j¦	&	\GR{j}{062C}&\GG{j}&\GG{j}&\GG{j}&\GG{j}&\GN[turk]{j}{0698}&\GN[kurdish]{j}{0698}&\GG{j}&\GG{j}&\GG{j} \\
% ¦:j¦	&	&&&&\GR[sindhi]{:j}{0684}&&&&&\\ ^^A(sindhi) 
% ¦jh¦	&	&&&&\GN[sindhi]{jh}{06A9}&&&&&\\
% ¦k¦	&	\GR{k}{0643}&\GG{k}&\GG{k}&\GG{k}&\GN[sindhi]{k}{06AA}&\GG{k}&\GG{k}&\GG{k}&\GG{k}&\GG{k} \\
% ^^A¦K¦	&	\GN{K}{}&&&&&&&&&\\ 
% ¦.k¦	&	\GV{.k}{06A9}&&&&&\GN[turk]{.k}{0642}&&&&\\
% ¦_k¦	&	\GR{_k}{063A}&&&&&&&&&\\
% ¦kh¦	&	&&&&\GN[sindhi]{kh}{}&&&&&\\
% ¦l¦	&	\GR{l}{0644}&\GG{l}&\GG{l}&\GG{l}&\GG{l}&\GG{l}&\GG{l}&\GG{l}&\GG{l}&\GG{l} \\ 
% ¦.l¦	&       &&&&&&\GR[kurdish]{.l}{06B6*}&&&\\ 
% ¦^l¦	&       &&&&&&\GR[kurdish]{^l}{06B5}&&&\\^^A(kurdish)  
% ¦m¦	&	\GR{m}{0645}&\GG{m}&\GG{m}&\GG{m}&\GG{m}&\GG{m}&\GG{m}&\GG{m}&\GG{m}&\GG{m} \\  
% ^^A¦M¦	&	\GN{M}{}&&&&&&&&\GN[malay]{M}{}&\\
% {\tiny ¦.mIN¦}	& 	&&&&\GN[sindhi]{.mIN}{06FE}&&&&&\\ 
% {\tiny ¦'|IN¦}	&	&&&&\GN[sindhi]{'|IN}{06FD}&&&&&\\ 
% ¦n¦	&	\GR{n}{0646}&\GG{n}&\GG{n}&\GG{n}&\GG{n}&\GG{n}&\GG{n}&\GG{n}&\GG{n}&\GG{n} \\ 
% ¦aN¦	&	\GR{BaN}{064B}&&&&&&&&&\\
% ¦uN¦	&	\GR{BuN}{064C}&&&&&&&&&\\ 
% ¦iN¦	&	\GR{BiN}{064D}&&&&&&&&&\\
% ¦.n¦	&	 &&\GR{.n}{06BA}&&&&&&&\\
% ¦..n¦	&       &&&&\GB[sindhi]{..n}{06B2*}&&&&&\\
% ¦,n¦	&	&&&\GR[pashto]{,n}{06BC}&\GN[sindhi]{,n}{06BB}&&&&&\\
% ¦^n¦	&	&&&&\GN[sindhi]{^n}{0683}&&&&\GN[malay]{^n}{06BD}&\GR[uighur]{^n}{06AD} \\
% ¦:n¦	&	&&&&\GR[sindhi]{:n}{06B1}&&&&&\\
% ¦o¦	&	&\GR[farsi]{Bo}{}&&\GN[pashto]{Bo}{0657}&&&\GN[kurdish]{o/Bo}{}&\GN[kashmiri]{Bo}{06C6}&&\GN[uighur]{Bo}{} \\
% ¦O¦	&		&\GN[farsi]{BO}{}&\GN[urdu]{BO}{}&\GN[pashto]{BO}{}&&&\GN[kurdish]{O/BO}{}&\GN[kashmiri]{BO}{}&&\\
% ¦ao¦	&	&&\GR[urdu]{Bao}{}&&\GR[sindhi]{Bao}{}&&&&&\\  
% ¦.o¦	&	&&&&&&&\GN[kashmiri]{B.o}{06C4}&&\\
% ¦.O¦	&	&&&&&&&\GN[kashmiri]{B.O}{}&&\\   
% ¦_o¦	&	&\GR{B_o}{}&&&&&&&&\\ 
% ¦_O¦	&	&\GR{B_O}{}&&&&&&&&\\
% ¦:o¦	&	&&&&&&&&&\GN[uighur]{:o/B:o}{06C6} \\
% ¦:O¦	&	\GV{:O}{06FC}&&&&&&&&&\\  
% ¦p¦	&	&\GR{p}{067E}&\GG{p}&\GG{p}&\GG{p}&\GG{p}&\GG{p}&\GG{p}&\GN[malay]{p}{06A8}&\GG{p} \\
% ^^A¦.p¦	&	\GN{.p}{}	&&&&&&&&&\\
% ¦ph¦	&	&&&&\GN[sindhi]{ph}{06A6}&&&&&\\
% ¦q¦	&	\GR{q}{0642}&\GG{q}&\GG{q}&\GG{q}&\GG{q}&\GG{q}&\GG{q}&\GG{q}&\GG{q}&\GG{q} \\ ^^A\GN[maghribi]{q}{} 
% ¦.q¦	&	\GV{.q}{066F}&&&&&&&&&\\  
% ¦r¦	&	\GR{r}{0631}&\GG{r}&\GG{r}&\GG{r}&\GG{r}&\GG{r}&\GG{r}&\GG{r}&\GG{r}&\GG{r} \\  
% ¦.r¦	&	\GV{.r}{0694*}&&&&&&\GN[kurdish]{.r}{0695}&&&\\ 
% ¦,r¦	&	&&\GR[urdu]{,r}{0691}&\GN[pashto]{,r}{0693}&\GN[sindhi]{,r}{0699}&&\GN[kurdish]{,r}{0694*}&&&\\
% ¦^r¦	&	\GV{^r}{06EF*}&&&&&&\GN[kurdish]{^r}{0692*}&&&\\
% ¦:r¦	&	\GV{:r}{0697*}$^c$&&&&&&&&&\\ ^^ADargwa in Dagestan
% ¦s¦	&	\GR{s}{0633}&\GG{s}&\GG{s}&\GG{s}&\GG{s}&\GG{s}&\GG{s}&\GG{s}&\GG{s}&\GG{s} \\ 
% ^^A¦S¦	&	\GN{S}{}&&&&&&&&&\\ 
% ¦.s¦	&	\GR{.s}{0635}&\GG{.s}&\GG{.s}&\GG{.s}&\GG{.s}&\GG{.s}&\GG{.s}&\GG{.s}&\GG{.s}&\\ 
% ¦,s¦	&	&&&\GR[pashto]{,s}{069A}&&\GN[turk]{,s}{0634}&&&&\\
% ¦^s¦	&	\GR{^s}{0634}&\GG{^s}&\GG{^s}&\GG{^s}&\GG{^s}&\GG{^s}&\GG{^s}&\GG{^s}&\GG{^s}&\GG{^s} \\  
% ¦_s¦	&	&&&&\GR{_s}{062B}&&&&&\\ 
% ¦:s¦	&	\GV{:s}{069B}&&&&&&&&&\\
% ¦t¦	&	\GR{t}{062A}&\GG{t}&\GG{t}&\GG{t}&\GG{t}&\GG{t}&\GG{t}&\GG{t}&\GG{t}&\GG{t} \\ 
% ¦T¦	&	\GR{T}{062A}&&&&&&&\GG{T}&&\\
% ¦.t¦	&	\GR{.t}{0637}&\GG{.t}&\GG{.t}&\GG{.t}&\GG{.t}&\GG{.t}&\GG{.t}&\GG{.t}&\GG{.t}&\\
% ^^A¦.T¦	&	\GN{.T}{}&&&&&&&&&\\  
% ¦,t¦	&	&&\GR[urdu]{,t}{0679}&\GN[pashto]{,t}{067C}&\GN[sindhi]{,t}{067D}&&&&&\\
% ¦_t¦	&	\GR{_t}{062B}&\GG{_t}&\GG{_t}&&\GG{_t}&\GG{_t}&\GG{_t}&\GG{_t}&\GG{_t}&\\  
% ^^A¦:t¦	&	\GN{:t}{}&&&&&&&&&\\  
% ¦th¦	&	&&&&\GN[sindhi]{th}{067F}&&&&&\\
% ¦,th¦	&	&&&&\GN[sindhi]{,th}{067A}&&&&&\\
% ¦u¦	&	\GR{Bu}{064F}&\GG{Bu}&\GG{Bu}&\GG{Bu}&\GG{Bu}&\GG{Bu}&\GN[kurdish]{u/Bu}{}&\GG{Bu}&\GG{Bu}&\GN[uighur]{u/Bu}{06C7} \\  
% ¦U¦	&	\GR{BU}{}&\GG{BU}&\GG{BU}&\GG{BU}&\GG{BU}&\GG{BU}&\GN[kurdish]{U/BU}{}&\GN[kashmiri]{BU}{0648+0657}&\GG{BU}&\\ 
% ¦.u¦	&	&&&&&&&\GN[kashmiri]{B.u}{0655}&&\\
% ¦.U¦	&	&&&&&&&\GN[kashmiri]{B.U}{0673}&&\\  
% ^^A¦^u¦	&	\GN{B^u}{}&&&&&&&&&\\   
% ¦_u¦	&	\GR{B_u}{0657}&&&&&&&&&\\  
% ^^A¦_U¦	&	\GN{B_U}{}&&&&&&&&&\\ 
% ¦:u¦	&       &&&&&&&&&\GN[uighur]{:u/B:u}{06C8} \\
% ¦:U¦	&	\GV{:U}{06C7}$^d$&&&&&&&&&\\ 
% ¦v¦	&	\GR{v}{06A4}$^e$&&&&&&&&\GN[malay]{v}{06CF}&\\ ^^A\GN[maghribi]{v}{06A5} 
% ^^A¦V¦	&	\GN{V}{}&&&&&&&&&\\ 
% ^^A¦.v¦	&	\GN{.v}{}&&&&&&&&&\\ 
% ¦w¦	&	\GR{w}{0648}&\GG{w}&\GG{w}&\GG{w}&\GG{w}&\GG{w}&\GG{w}&\GG{w}&\GG{w}&\GN[uighur]{w}{06CB} \\
% ¦W¦	&	\GR{W}{}&&&&&&&&&\\  
% ¦^w¦	&	\GV{^w}{06C9*}&&&&&&&&&\\
% ¦:w¦	&	\GV{:w}{06CA*}&&&&&&&&&\\
% ¦x¦	&	\GR{x}{062E}&\GG{x}&\GG{x}&\GG{x}&\GG{x}&\GG{x}&\GG{x}&\GG{x}&\GG{x}&\GN[uighur]{x}{} \\ 
% ¦y¦	&	\GR{y}{064A}&\GN[farsi]{y}{06CC}&\GG[urdu]{y}&\GG[pashto]{y}&\GG{y}&\GG[turk]{y}&\GG{y}&\GG[kashmiri]{y}&\GG{y}&\GG{y} \\  
% ¦Y¦	&	\GR{"Y}{0649}&&&&&&&&&\\  
% ¦.y¦	&       &&&&&&&\GN[kashmiri]{B.yB}{}&&\\  
% ¦z¦	&	\GR{z}{0632}&\GG{z}&\GG{z}&\GG{z}&\GG{z}&\GG{z}&\GG{z}&\GG{z}&\GG{z}&\GG{z} \\ 
% ¦.z¦	&	\GR{.z}{0638}&\GG{.z}&\GG{.z}&\GG{.z}&\GG{.z}&\GG{.z}&\GG{.z}&\GG{.z}&\GG{.z}&\\ 
% ¦,z¦	&	 &&&\GR[pashto]{,z}{0696}&&\GN[turk]{,z}{0636}&&&&\\
% ¦^z¦	&	 &\GR{^z}{0698}&\GG{^z}&\GG{^z}&&&&\GG{^z}&&\GG{^z} \\ 
% ¦_z¦	&	\GR{_z}{0630}$^f$&&&&&&&&&\\
% ¦:z¦	&	&&&&&&&\GN[kashmiri]{:z}{0636}&&\\ 
% ¦'¦	&	\GR{'}{0621}&\GG{'}&\GG{'}&\GG{'}&\GG{'}&\GG{'}&\GG{'}&\GG{'}&\GG{'}&\GG{ئ-} \\
% ¦`¦	&	\GR{`}{0639}&\GG{`}&\GG{`}&\GG{`}&\GG{`}&\GG{`}&\GG{`}&\GG{`}&\GG{`}&\\
% &&&&&&&&&& \\
% \end{supertabular}
% \end{center}
% \footnotesize
% 
% $^a$ For Western Punjabi (Lahnda).\\
% $^b$ Alternative form of \textmalay{g} in Malay.\\
% $^c$ For Dargwa (language of Dagestan).\\
% $^d$ For Kirgiz (and Uighur).\\
% $^e$ To transliterate dialects and foreign words.\\
% $^f$ Alternative to ¦_d¦.\\
% \normalsize\bigskip
% 
% Maghribi Arabic is identical to Arabic except for the three letters ¦f¦, ¦q¦
% and ¦v¦ which yield the glyphs \textmaghribi{f} (\textsf{U+06A2}), \textmaghribi{q}
% (\textsf{U+06A7}), and \textmaghribi{v} (\textsf{U+06A5}), respectively.
% 
% ^^A TODO : implement the following ??
% ^^A perhaps unnecessary coz it can be coded by means of Urdu
% ^^A yet it might be nice to code pashto in a standard way and 
% ^^A be able to switch the representation to Pakistani
% ^^A Differences between standard Pashto and Pakistani Pashto (from inspection of the Omega OTP files)
% ^^A064A -> 06D2
% ^^A069A -> 062E ??
% ^^A06A9 -> 0643
% ^^A06AF -> 06AB
% ^^A06CD -> 06D2
% ^^A06D0 -> 06D2
% ^^A0626 -> 06D2
% ^^A0649 -> 06D2
% ^^A067C -> 0679
% ^^A0693 -> 0691
% ^^A0696 -> 06AF
% ^^A06BC -> 0646 0631
% ^^A0689 -> 0688
% ^^A
% 
% \section{Unicode-Encoding concordance}
% \parindent 0pt
% \let\oldbaselineskip=\baselineskip
% \baselineskip=20pt
% \begin{multicols}{2}
% \begin{tabbing}
% \textsf{060B} \qquad \=  {\arabicfont\char"060B} \qquad \= \kill
% \textsf{060B}   \>   {\arabicfont\char"060B} \\
% \textsf{060C}   \>   {\arabicfont\char"060C} \>	¦,¦ \\
% \textsf{060D}   \>   {\arabicfont\char"060D} \\
% \textsf{060E}   \>   {\arabicfont\char"060E} \\
% \textsf{060F}   \>   {\arabicfont\char"060F} \\
% \textsf{0610}   \>   {\arabicfont\char"0610} \\
% \textsf{0611}   \>   {\arabicfont\char"0611} \\
% \textsf{0612}   \>   {\arabicfont\char"0612} \\
% \textsf{0613}   \>   {\arabicfont\char"0613} \\
% \textsf{0614}   \>   {\arabicfont\char"0614} \\
% \textsf{0615}   \>   {\arabicfont\char"0615} \\
% ^^AU+0616\\ %\textarab{\char"0616} \\
% ^^AU+0617\\ %\textarab{\char"0617} \\
% ^^AU+0618\\ %\textarab{\char"0618} \\
% ^^AU+0619\\ %\textarab{\char"0619} \\
% ^^AU+061A\\ %\textarab{\char"061A} \\
% \textsf{061B}   \>   {\arabicfont\char"061B}	\>	¦;¦    \\ ^^A\textarab{ ; }   \\
% ^^AU+061C\\ %\textarab{\char"061C} \\
% ^^AU+061D\\ %\textarab{\char"061D} \\
% \textsf{061E}   \>   {\arabicfont\char"061E}	\>	¦::¦/¦DOTS¦ \\
% \textsf{061F}   \>   {\arabicfont\char"061F}	\>	¦?¦    \\ ^^A\textarab{ ? }   \\
% ^^AU+0620\\ %\textarab{\char"0620} \\
% \textsf{0621}   \>   {\arabicfont\char"0621}	\>	¦|"'¦  \textit{or context}  \\ ^^A\textverb{ '| }   \\
% \textsf{0622}   \>   {\arabicfont\char"0622}	\>	¦A"'¦  \textit{or context}  \\ ^^A\textverb{ 'A }   \\
% \textsf{0623}   \>   {\arabicfont\char"0623}	\>	¦a"'¦  \textit{or context}  \\ ^^A\textverb{ 'a }   \\
% \textsf{0624}   \>   {\arabicfont\char"0624}	\>	¦w"'¦  \textit{or context}  \\ ^^A\textverb{ 'w }   \\
% \textsf{0625}   \>   {\arabicfont\char"0625}	\>	¦i"'¦  \textit{or context}  \\ ^^A\textverb{ 'i }   \\
% \textsf{0626}   \>   {\arabicfont\char"0626}	\>	¦y"'¦  \textit{or context}  \\ ^^A\textverb{ 'y }   \\
% \textsf{0627}   \>   {\arabicfont\char"0627}	\>	¦A¦    \\ ^^A\textarab{ A }   \\
% \textsf{0628}   \>   {\arabicfont\char"0628}	\>	¦b¦    \\ ^^A\textarab{ b }   \\
% \textsf{0629}   \>   {\arabicfont\char"0629}	\>	¦T¦    \\ ^^A\textarab{ T }   \\
% \textsf{062A}   \>   {\arabicfont\char"062A}	\>	¦t¦    \\ ^^A\textarab{ t }   \\
% \textsf{062B}   \>   {\arabicfont\char"062B}	\>	¦_t¦    \\ ^^A\textarab{ _t } \\
% \textsf{062C}   \>   {\arabicfont\char"062C}	\>	¦^g¦/¦j¦    \\ ^^A\textarab{ j } \\
% \textsf{062D}   \>   {\arabicfont\char"062D}	\>	¦.h¦    \\ ^^A\textarab{ .h }   \\
% \textsf{062E}   \>   {\arabicfont\char"062E}	\>	¦_h¦/¦x¦    \\ ^^A\textarab{ x } \\
% \textsf{062F}   \>   {\arabicfont\char"062F}	\>	¦d¦    \\ ^^A\textarab{ d }   \\
% \textsf{0630}   \>   {\arabicfont\char"0630}	\>	¦_d¦    \\ ^^A\textarab{ _d }  \\
% \textsf{0631}   \>   {\arabicfont\char"0631}	\>	¦r¦    \\ ^^A\textarab{ r }   \\
% \textsf{0632}   \>   {\arabicfont\char"0632}	\>	¦z¦    \\ ^^A\textarab{ z }   \\
% \textsf{0633}   \>   {\arabicfont\char"0633}	\>	¦s¦    \\ ^^A\textarab{ s }   \\
% \textsf{0634}   \>   {\arabicfont\char"0634}	\>	¦^s¦    \\ ^^A\textarab{ ^s }  \\
% \textsf{0635}   \>   {\arabicfont\char"0635}	\>	¦.s¦    \\ ^^A\textarab{ .s }   \\
% \textsf{0636}   \>   {\arabicfont\char"0636}	\>	¦.d¦    \\ ^^A\textarab{ .d } \\
% \textsf{0637}   \>   {\arabicfont\char"0637}	\>	¦.t¦    \\ ^^A\textarab{ .t } \\
% \textsf{0638}   \>   {\arabicfont\char"0638}	\>	¦.z¦    \\ ^^A\textarab{ .z } \\
% \textsf{0639}   \>   {\arabicfont\char"0639}	\>	¦`¦    \\ ^^A\textarab{ ` } \\
% \textsf{063A}   \>   {\arabicfont\char"063A}	\>	¦.g¦    \\ ^^A\textarab{ .g } \\
% \textsf{0640}   \>   {\arabicfont\char"0640}	\>	¦B¦    \\ ^^A\textarab{ B } \\
% \textsf{0641}   \>   {\arabicfont\char"0641}	\>	¦f¦    \\ ^^A\textarab{ f } \\
% \textsf{0642}   \>   {\arabicfont\char"0642}	\>	¦q¦    \\ ^^A\textarab{ q } \\
% \textsf{0643}   \>   {\arabicfont\char"0643}	\>	¦k¦    \\ ^^A\textarab{ k } \\
% \textsf{0644}   \>   {\arabicfont\char"0644}	\>	¦l¦    \\ ^^A\textarab{ l } \\
% \textsf{0645}   \>   {\arabicfont\char"0645}	\>	¦m¦    \\ ^^A\textarab{ m } \\
% \textsf{0646}   \>   {\arabicfont\char"0646}	\>	¦n¦    \\ ^^A\textarab{ n } \\
% \textsf{0647}   \>   {\arabicfont\char"0647}	\>	¦h¦    \\ ^^A\textarab{ h } \\
% \textsf{0648}   \>   {\arabicfont\char"0648}	\>	¦w¦    \\ ^^A\textarab{ w } \\
% \textsf{0649}   \>   {\arabicfont\char"0649}	\>	¦Y¦    \\ ^^A\textarab{ Y } \\
% \textsf{064A}   \>   {\arabicfont\char"064A}	\>	¦y¦    \\ ^^A\textarab{ y } \\
% \textsf{064B}   \>   {\arabicfont\char"064B}	\>	¦aN¦    \\ ^^A\textarab{ |BaN } \\
% \textsf{064C}   \>   {\arabicfont\char"064C}	\>	¦uN¦    \\ ^^A\textarab{ |BuN } \\
% \textsf{064D}   \>   {\arabicfont\char"064D}	\>	¦iN¦    \\ ^^A\textarab{ |BiN } \\
% \textsf{064E}   \>   {\arabicfont\char"064E}	\>	¦a¦    \\ ^^A\textarab{ |Ba } \\
% \textsf{064F}   \>   {\arabicfont\char"064F}	\>	¦u¦    \\ ^^A\textarab{ |Bu } \\
% \textsf{0650}   \>   {\arabicfont\char"0650}	\>	¦i¦    \\ ^^A\textarab{ |Bi } \\
% \textsf{0651}   \>   {\arabicfont\char"0651}\\ ^^A\textarab{ |BB }\\
% \textsf{0652}   \>   {\arabicfont\char"0652}	\>	¦"¦ \textit{after a consonant or} ¦°¦ \\
% \textsf{0653}   \>   {\arabicfont\char"0653}	\>	¦^U¦   \\
% \textsf{0654}   \>   {\arabicfont\char"0654}	\>	¦'B¦    \\ ^^A\textverb{ 'B } \\
% \textsf{0655}   \>   {\arabicfont\char"0655}	\>	¦.u¦  \textit{kashmiri}  \\
% \textsf{0656}   \>   {\arabicfont\char"0656}	\>	¦I¦  \textit{kashmiri}  \\
% \textsf{0657}   \>   {\arabicfont\char"0657}	\>	¦o¦  \textit{pashto}  \\ ^^A\textpashto{ Bo }  \\
% \textsf{0658}   \>   {\arabicfont\char"0658}	\>	\textit{see kashmiri} ¦e¦   \\
% \textsf{0659}   \>   {\arabicfont\char"0659}	\>	¦e¦  \textit{pashto}  \\
% \textsf{065A}   \>   {\arabicfont\char"065A}   \\
% \textsf{065B}   \>   {\arabicfont\char"065B}   \\
% \textsf{065C}   \>   {\arabicfont\char"065C}   \\
% \textsf{065D}   \>   {\arabicfont\char"065D}   \\
% \textsf{065E}   \>   {\arabicfont\char"065E}   \\
% \textsf{0660}   \>   {\arabicfont\char"0660}	\>	¦0¦    \\ ^^A\textarab{ 0 } \\
% \textsf{0661}   \>   {\arabicfont\char"0661}	\>	¦1¦    \\ ^^A\textarab{ 1 } \\
% \textsf{0662}   \>   {\arabicfont\char"0662}	\>	¦2¦    \\ ^^A\textarab{ 2 } \\
% \textsf{0663}   \>   {\arabicfont\char"0663}	\>	¦3¦    \\ ^^A\textarab{ 3 } \\
% \textsf{0664}   \>   {\arabicfont\char"0664}	\>	¦4¦    \\ ^^A\textarab{ 4 } \\
% \textsf{0665}   \>   {\arabicfont\char"0665}	\>	¦5¦    \\ ^^A\textarab{ 5 } \\
% \textsf{0666}   \>   {\arabicfont\char"0666}	\>	¦6¦    \\ ^^A\textarab{ 6 } \\
% \textsf{0667}   \>   {\arabicfont\char"0667}	\>	¦7¦    \\ ^^A\textarab{ 7 } \\
% \textsf{0668}   \>   {\arabicfont\char"0668}	\>	¦8¦    \\ ^^A\textarab{ 8 } \\
% \textsf{0669}   \>   {\arabicfont\char"0669}	\>	¦9¦    \\ ^^A\textarab{ 9 } \\
% \textsf{066A}   \>   {\arabicfont\char"066A}	\>	¦\^^A¦    \\ %\textarab{ \% } \\
% \textsf{066B}   \>   {\arabicfont\char"066B}	\>	    \\ ^^A\textarab{ , }  \\
% \textsf{066C}   \>   {\arabicfont\char"066C}	\>	¦,¦    \\ ^^A\textarab{ , }  \\
% \textsf{066D}   \>   {\arabicfont\char"066D}	\>	¦*¦    \\ ^^A\textarab{ * }  \\
% \textsf{066E}   \>   {\arabicfont\char"066E}	\>	¦.b¦, \quad ¦i¦  \textit{uighur} \\ ^^A\textarab{ .b }  \\
% \textsf{066F}   \>   {\arabicfont\char"066F}	\>	¦.f¦    \\ ^^A\textarab{ .f }  \\
% \textsf{0670}   \>   {\arabicfont\char"0670}	\>	¦_a¦    \\ ^^A\textarab{ B_a }  \\
% \textsf{0671}   \>   {\arabicfont\char"0671}	\>	\textit{automatic}    \\
% \textsf{0672}   \>   {\arabicfont\char"0672}	\>	¦.A¦  \textit{kashmiri}  \\
% \textsf{0673}   \>   {\arabicfont\char"0673}	\>	¦.U¦  \textit{kashmiri}  \\
% \textsf{0674}   \>   {\arabicfont\char"0674}   \\
% \textsf{0675}   \>   {\arabicfont\char"0675}   \\
% \textsf{0676}   \>   {\arabicfont\char"0676}   \\
% \textsf{0677}   \>   {\arabicfont\char"0677}   \\
% \textsf{0678}   \>   {\arabicfont\char"0678}   \\
% \textsf{0679}   \>   {\arabicfont\char"0679}	\>	¦,t¦    \\ ^^A\textarab{ ,t } \\
% \textsf{067A}   \>   {\arabicfont\char"067A}	\>	¦,th¦    \\ ^^A\textsindhi{ ,th }  \\
% \textsf{067B}   \>   {\arabicfont\char"067B}	\>	¦:b¦    \\ ^^A\textsindhi{ :b }  \\
% \textsf{067C}   \>   {\arabicfont\char"067C}	\>	¦,t¦    \\ ^^A\textpashto{ ,t }  \\
% \textsf{067D}   \>   {\arabicfont\char"067D}	\>	¦,t¦    \\ ^^A\textsindhi{ ,t }  \\
% \textsf{067E}   \>   {\arabicfont\char"067E}	\>	¦p¦    \\ ^^A\textarab{ p }  \\
% \textsf{067F}   \>   {\arabicfont\char"067F}	\>	¦th¦    \\ ^^A\textsindhi{ th }  \\
% \textsf{0680}   \>   {\arabicfont\char"0680}	\>	¦bh¦    \\ ^^A\textsindhi{ bh } \\
% \textsf{0681}   \>   {\arabicfont\char"0681}	\>	¦c¦    \\ ^^A\textarab{ c }  \\
% \textsf{0682}   \>   {\arabicfont\char"0682}	\>	¦:c¦   \\ ^^A (old pashto)
% \textsf{0683}   \>   {\arabicfont\char"0683}	\>	¦^n¦    \\ ^^A\textsindhi{ ^n }  \\
% \textsf{0684}   \>   {\arabicfont\char"0684}	\>	¦:j¦    \\ ^^A\textarab{ :j }  %sindhi\\
% \textsf{0685}   \>   {\arabicfont\char"0685}	\>	¦,c¦    \\ ^^A\textarab{ ,c }  %pashto\\
% \textsf{0686}   \>   {\arabicfont\char"0686}	\>	¦^c¦, \quad ¦c¦  \textit{malay} \\ ^^A\textarab{ ^c }  %farsi+\\
% \textsf{0687}   \>   {\arabicfont\char"0687}	\>	¦^ch¦  \textit{sindhi}  \\ ^^A\textsindhi{ ^ch }  \\
% \textsf{0688}   \>   {\arabicfont\char"0688}	\>	¦,d¦    \\ ^^A\textarab{ ,d }   %urdu\\
% \textsf{0689}   \>   {\arabicfont\char"0689}	\>	¦,d¦  \textit{pashto}  \\ ^^A\textpashto{ ,d }  \\
% \textsf{068A}   \>   {\arabicfont\char"068A}	\>	¦,d¦  \textit{sindhi}  \\ ^^A\textsindhi{ ,d }  \\
% \textsf{068B}   \>   {\arabicfont\char"068B}	\>	¦.,d¦  \\ ^^A(Western punjabi)
% \textsf{068C}   \>   {\arabicfont\char"068C}	\>	¦dh¦  \textit{sindhi}  \\ ^^A\textsindhi{ dh }  \\
% \textsf{068D}   \>   {\arabicfont\char"068D}	\>	¦,dh¦  \textit{sindhi}  \\ ^^A\textsindhi{ ,dh }  \\
% \textsf{068E}   \>   {\arabicfont\char"068E}	\>	¦^d¦  \textit{sindhi}  \\
% \textsf{068F}   \>   {\arabicfont\char"068F}	\>	¦:d¦    \\ ^^A\textarab{ :d }  \\
% \textsf{0690}   \>   {\arabicfont\char"0690}	\>	¦::d¦  \textit{urdu}  \\
% \textsf{0691}   \>   {\arabicfont\char"0691}	\>	¦,r¦    \\ ^^A\textarab{ ,r }  \\
% \textsf{0692}   \>   {\arabicfont\char"0692}	\>	¦^r¦  \textit{kurdish}    \\ ^^ANEW-> ^r\\
% \textsf{0693}   \>   {\arabicfont\char"0693}	\>	¦,r¦  \textit{pashto}  \\ ^^A\textpashto{ ,r }  \\
% \textsf{0694}   \>   {\arabicfont\char"0694}	\>	¦.r¦, \quad ¦,r¦  \textit{kurdish}  \\ ^^ANEW-> kurdish ,r\\
% \textsf{0695}   \>   {\arabicfont\char"0695}	\>	¦.r¦  \textit{kurdish}   \\ ^^A\textarab{ .r} \\
% \textsf{0696}   \>   {\arabicfont\char"0696}	\>	¦,z¦    \\ ^^A\textarab{ ,z }  \\
% \textsf{0697}   \>   {\arabicfont\char"0697}	\>	¦:r¦  \\
% \textsf{0698}   \>   {\arabicfont\char"0698}	\>	¦^z¦, \quad ¦j¦  \textit{ottoman} \\ ^^A\textarab{ ^z }  \\
% \textsf{0699}   \>   {\arabicfont\char"0699}	\>	¦:s¦  \\ ^^A\textarab{ :s }  \\
% \textsf{069A}   \>   {\arabicfont\char"069A}	\>	¦,s¦  \\
% \textsf{069B}   \>   {\arabicfont\char"069B}	\>	¦:s¦  \\ 
% \textsf{069C}   \>   {\arabicfont\char"069C} \\
% \textsf{069D}   \>   {\arabicfont\char"069D} \\
% \textsf{069E}   \>   {\arabicfont\char"069E} \\
% \textsf{069F}   \>   {\arabicfont\char"069F} \\
% \textsf{06A0}   \>   {\arabicfont\char"06A0}	\>	¦^g¦  \textit{ottoman+malay}  \\ ^^A\textmalay{ ^g }  \\
% \textsf{06A1}   \>   {\arabicfont\char"06A1}	\>	¦.f¦   \\ ^^A\textarab{ .f }  \\
% \textsf{06A2}   \>   {\arabicfont\char"06A2}	\>	¦f¦    \\ ^^A\textarab{ f }  \\
% \textsf{06A3}   \>   {\arabicfont\char"06A3}   \\
% \textsf{06A4}   \>   {\arabicfont\char"06A4}	\>	¦v¦, \quad ¦p¦  \textit{malay}  \\ ^^A\textarab{ V }  \\
% \textsf{06A5}   \>   {\arabicfont\char"06A5}	\>	¦v¦   \textit{maghribi}  \\ ^^A\textarab{ v }  \\
% \textsf{06A6}   \>   {\arabicfont\char"06A6}	\>	¦ph¦  \textit{sindhi}  \\ ^^A\textarab{ ph }  \\
% \textsf{06A7}   \>   {\arabicfont\char"06A7}	\>	¦q¦    \\ ^^A\textarab{ q }  \\
% \textsf{06A8}   \>   {\arabicfont\char"06A8}	\>	¦p¦    \\ ^^A\textarab{ p }  \\
% \textsf{06A9}   \>   {\arabicfont\char"06A9}	\>	¦.k¦    \\ ^^A\textarab{ .k }   % also M\\
% \textsf{06AA}   \>   {\arabicfont\char"06AA}	\>	¦k¦    \\ ^^A\textarab{ k }  \\
% \textsf{06AB}   \>   {\arabicfont\char"06AB}	\>	¦G¦    \\ ^^A\textarab{ G }  \\
% \textsf{06AC}   \>   {\arabicfont\char"06AC}	\>	¦,g¦    \\ ^^A\textarab{ ,g }  \\
% \textsf{06AD}   \>   {\arabicfont\char"06AD}	\>	¦^n¦    \\ ^^A\textarab{ ^n }   % also K\\
% \textsf{06AE}   \>   {\arabicfont\char"06AE}   \\
% \textsf{06AF}   \>   {\arabicfont\char"06AF}	\>	¦g¦    \\ ^^A\textarab{ g }  \\
% \textsf{06B0}   \>   {\arabicfont\char"06B0}   \\
% \textsf{06B1}   \>   {\arabicfont\char"06B1}	\>	¦:n¦    \\ ^^A\textarab{ :n }  \\
% \textsf{06B2}   \>   {\arabicfont\char"06B2}	\>	¦..n¦  \textit{sindhi}  \\
% \textsf{06B3}   \>   {\arabicfont\char"06B3}	\>	¦:g¦    \\ ^^A\textarab{ :g }  \\
% \textsf{06B4}   \>   {\arabicfont\char"06B4}	\>	¦.:g¦  \textit{sindhi}  \\
% \textsf{06B5}   \>   {\arabicfont\char"06B5}	\>	¦^l¦    \\ ^^A\textarab{ ^l }   % also .l -> prob a bug for the next glyph\\
% \textsf{06B6}   \>   {\arabicfont\char"06B6}	\>	¦.l¦     \\ ^^ANEW -> .l\\
% \textsf{06B7}   \>   {\arabicfont\char"06B7}     \\ ^^ANEW -    \\ %NEW -> :l U+0651 ;;SHADDA\\
% \textsf{06B8}   \>   {\arabicfont\char"06B8}   \\
% \textsf{06B9}   \>   {\arabicfont\char"06B9}   \\
% \textsf{06BA}   \>   {\arabicfont\char"06BA}	\>	¦.n¦ \\ ^^A\textarab{ .n }  \\
% \textsf{06BB}   \>   {\arabicfont\char"06BB}	\>	¦,n¦  \textit{sindhi}  \\ ^^A\textsindhi{ .n }  \\
% \textsf{06BC}   \>   {\arabicfont\char"06BC}	\>	¦,n¦ \\ ^^A\textpashto{ .n }  \\
% \textsf{06BD}   \>   {\arabicfont\char"06BD}	\>	¦^n¦  \textit{malay}  \\ ^^A\textmalay{ ny } %??? with \setmalay ^n ArabTeX produces nun with 3 dots below, but this is not in Unicode\\
% \textsf{06BE}   \>   {\arabicfont\char"06BE}	\>	¦h¦  \textit{urdu}  \\ ^^A\texturdu{ h }  \\
% \textsf{06BF}   \>   {\arabicfont\char"06BF}	\>	¦.^c¦  \\ ^^ANEW -> .^c\\
% \textsf{06C0}   \>   {\arabicfont\char"06C0}	\>	¦h"'¦  \textit{or context}  \\ ^^A\textverb{ 'h }  %verbatim ???\\
% \textsf{06C1}   \>   {\arabicfont\char"06C1}	\>	¦,h¦ \\
% \textsf{06C2}   \>   {\arabicfont\char"06C2}	\>	¦H-¦ (\textit{+ ezafe})\footnote{ Or is it ¦,h¦+ezafe?} \textit{urdu} \\ 
% \textsf{06C3}   \>   {\arabicfont\char"06C3}	\>	¦H¦  \textit{urdu}  \\ ^^A\texturdu{ T }  %???\\
% \textsf{06C4}   \>   {\arabicfont\char"06C4}	\>	¦.o¦ \textit{kashmiri}  \\ ^^A\textkashmiri{ |.o }  \\
% \textsf{06C5}   \>   {\arabicfont\char"06C5}   		\\
% \textsf{06C6}   \>   {\arabicfont\char"06C6}	\>	¦o¦ \textit{kashmiri} ,\quad ¦:o¦  \textit{uighur} \\ ^^A\textkashmiri{ |o }  \\
% \textsf{06C7}   \>   {\arabicfont\char"06C7}	\>	¦:U¦, \quad ¦u¦  \textit{uighur}  \\ ^^A\textuighur{ u }  \\
% \textsf{06C8}   \>   {\arabicfont\char"06C8}	\>	¦:u¦  \textit{uighur}  \\ ^^A\textuighur{ |:u }  \\
% \textsf{06C9}   \>   {\arabicfont\char"06C9}   	\>	¦^w¦ \\ ^^AArabTeX internals \a@roof\\
% \textsf{06CA}   \>   {\arabicfont\char"06CA}   	\>	¦:w¦ \\ ^^ANEW
% \textsf{06CB}   \>   {\arabicfont\char"06CB}	\>	¦w¦  \textit{uighur}  \\ ^^A\textuighur{ w }  \\
% \textsf{06CC}   \>   {\arabicfont\char"06CC}	\>	¦y¦/¦I¦  \textit{farsi, etc.},\\
%                 \>                              \>      ¦.I¦ \textit{arabic, sindhi, malay}  \\ 
% \textsf{06CD}   \>   {\arabicfont\char"06CD}	\>	¦Ee¦  \textit{pashto}  \\ ^^A\textpashto{\novocalize BEe }  \\
% \textsf{06CE}   \>   {\arabicfont\char"06CE}   \\
% \textsf{06CF}   \>   {\arabicfont\char"06CF}	\>	¦v¦  \textit{malay}  \\ ^^A\textmalay{ v }  \\
% \textsf{06D0}   \>   {\arabicfont\char"06D0}	\>	¦E¦  \textit{pashto}, \quad ¦e¦  \textit{uighur} \\ ^^A\textuighur{ |e } \\
% \textsf{06D1}   \>   {\arabicfont\char"06D1}   \\
% \textsf{06D2}   \>   {\arabicfont\char"06D2}	\>	¦E¦  \textit{urdu+kashmiri}  \\ ^^A\texturdu{ |E }  \\
% \textsf{06D3}   \>   {\arabicfont\char"06D3}	\>	¦'E¦  \textit{urdu}  \\ ^^A\texturdu{ |'E }  \\
% \textsf{06D4}   \>   {\arabicfont\char"06D4}	\>	¦..¦  \textit{urdu}  \\
% \textsf{06D5}   \>   {\arabicfont\char"06D5}   \\
% \textsf{06D6}   \>   {\arabicfont\char"06D6}	\>	¦^SLY¦   \\
% \textsf{06D7}   \>   {\arabicfont\char"06D7}	\>	¦^QLY¦   \\
% \textsf{06D8}   \>   {\arabicfont\char"06D8}	\>	¦^MIM¦   \\
% \textsf{06D9}   \>   {\arabicfont\char"06D9}	\>	¦^LA¦   \\
% \textsf{06DA}   \>   {\arabicfont\char"06DA}	\>	¦^JIM¦   \\
% \textsf{06DB}   \>   {\arabicfont\char"06DB}	\>	¦^DOTS¦   \\
% \textsf{06DC}   \>   {\arabicfont\char"06DC}	\>	¦^SIN¦   \\
% \textsf{06DD}   \>   {\arabicfont\char"06DD}	\>	¦[[<digits>]]¦   \\
% \textsf{06DE}   \>   {\arabicfont\char"06DE}	\>	¦HIZB¦   \\
% \textsf{06DF}   \>   {\arabicfont\char"06DF}	\>	¦CIRCZERO¦   \\
% \textsf{06E0}   \>   {\arabicfont\char"06E0}	\>	¦RECTZERO¦   \\
% \textsf{06E1}   \>   {\arabicfont\char"06E1}   \>	¦^JAZM¦  \\
% \textsf{06E2}   \>   {\arabicfont\char"06E2}   \>	¦^MIM¦   \\
% \textsf{06E3}   \>   {\arabicfont\char"06E3}   \>	¦_SIN¦   \\
% \textsf{06E4}   \>   {\arabicfont\char"06E4}   \>	¦^MADDA¦ \\
% \textsf{06E5}   \>   {\arabicfont\char"06E5}   \>	¦WAW¦    \\
% \textsf{06E6}   \>   {\arabicfont\char"06E6}   \>	¦YEH¦    \\
% \textsf{06E7}   \>   {\arabicfont\char"06E7}   \>	¦^YEH¦   \\
% \textsf{06E8}   \>   {\arabicfont\char"06E8}   \>	¦^NUN¦   \\
% \textsf{06E9}   \>   {\arabicfont\char"06E9}   \>	¦SAJDA¦   \\
% \textsf{06EA}   \>   {\arabicfont\char"06EA}   \>	¦_STOP¦  \\
% \textsf{06EB}   \>   {\arabicfont\char"06EB}   \>	¦^STOP¦  \\
% \textsf{06EC}   \>   {\arabicfont\char"06EC}   \>	¦^RSTOP¦ \\
% \textsf{06ED}   \>   {\arabicfont\char"06ED}   \>	¦_MIM¦   \\
% \textsf{06EE}   \>   {\arabicfont\char"06EE}   \>  	¦^d¦ \\
% \textsf{06EF}   \>   {\arabicfont\char"06EF}   \> 	¦^r¦ \\
% \textsf{06F0}   \>   {\arabicfont\char"06F0}   \>	¦0¦  \textit{persian, urdu, etc.}  \\
% \textsf{06F1}   \>   {\arabicfont\char"06F1}   \>	¦1¦  \textit{persian, urdu, etc.}  \\
% \textsf{06F2}   \>   {\arabicfont\char"06F2}   \>	¦2¦  \textit{persian, urdu, etc.}  \\
% \textsf{06F3}   \>   {\arabicfont\char"06F3}   \>	¦3¦  \textit{persian, urdu, etc.}  \\
% \textsf{06F4}   \>   {\arabicfont\char"06F4}   \>	¦4¦  \textit{persian, urdu, etc.}  \\
% \textsf{06F5}   \>   {\arabicfont\char"06F5}   \>	¦5¦  \textit{persian, urdu, etc.}  \\
% \textsf{06F6}   \>   {\arabicfont\char"06F6}   \>	¦6¦  \textit{persian, urdu, etc.}  \\
% \textsf{06F7}   \>   {\arabicfont\char"06F7}   \>	¦7¦  \textit{persian, urdu, etc.}  \\
% \textsf{06F8}   \>   {\arabicfont\char"06F8}   \>	¦8¦  \textit{persian, urdu, etc.}  \\
% \textsf{06F9}   \>   {\arabicfont\char"06F9}   \>	¦9¦  \textit{persian, urdu, etc.}  \\
% \textsf{06FA}   \>   {\arabicfont\char"06FA}   \\
% \textsf{06FB}   \>   {\arabicfont\char"06FB}   \\
% \textsf{06FC}   \>   {\arabicfont\char"06FC}	\>	¦:O¦   \\
% \textsf{06FD}   \>   {\arabicfont\char"06FD}	\>	¦.|IN¦  \textit{sindhi}  \\ ^^AArabTeX internal \a@bars\\
% \textsf{06FE}   \>   {\arabicfont\char"06FE}	\>	¦.MIN¦  \textit{sindhi}  \\
% \textsf{06FF}   \>   {\arabicfont\char"06FF}   \\
% \textbf{Arabic Supplement}   \\
% \textsf{0750}   \>   {\arabicfont\char"0750}   \\
% \textsf{0751}   \>   {\arabicfont\char"0751}   \\
% \textsf{0752}   \>   {\arabicfont\char"0752}   \\
% \textsf{0753}   \>   {\arabicfont\char"0753}   \\
% \textsf{0754}   \>   {\arabicfont\char"0754}   \\
% \textsf{0755}   \>   {\arabicfont\char"0755}   \\
% \textsf{0756}   \>   {\arabicfont\char"0756}   \\
% \textsf{0757}   \>   {\arabicfont\char"0757}   \\
% \textsf{0758}   \>   {\arabicfont\char"0758}   \\
% \textsf{0759}   \>   {\arabicfont\char"0759}   \\
% \textsf{075A}   \>   {\arabicfont\char"075A}   \\
% \textsf{075B}   \>   {\arabicfont\char"075B}   \\
% \textsf{075C}   \>   {\arabicfont\char"075C}   \\
% \textsf{075D}   \>   {\arabicfont\char"075D}   \\
% \textsf{075E}   \>   {\arabicfont\char"075E}   \\
% \textsf{075F}   \>   {\arabicfont\char"075F}   \\
% \textsf{0760}   \>   {\arabicfont\char"0760}   \\
% \textsf{0761}   \>   {\arabicfont\char"0761}   \\
% \textsf{0762}   \>   {\arabicfont\char"0762}	\>	¦g¦  \textit{malay}  \\ ^^A\textmalay{ g }  \\
% \textsf{0763}   \>   {\arabicfont\char"0763}   \\
% \textsf{0764}   \>   {\arabicfont\char"0764}   \\
% \textsf{0765}   \>   {\arabicfont\char"0765}   \\
% \textsf{0766}   \>   {\arabicfont\char"0766}   \\
% \textsf{0767}   \>   {\arabicfont\char"0767}   \\
% \textsf{0768}   \>   {\arabicfont\char"0768}   \\
% \textsf{0769}   \>   {\arabicfont\char"0769}   \\
% \textsf{076A}   \>   {\arabicfont\char"076A}   \\
% \textsf{076B}   \>   {\arabicfont\char"076B}   \\
% \textsf{076C}   \>   {\arabicfont\char"076C}   \\
% \textsf{076D}   \>   {\arabicfont\char"076D}   \\
% \textbf{Presentation Forms} * \\
% \textsf{FD3E}	\>   {\arabicfont\char"FD3E} 	\>	¦))¦ \\
% \textsf{FD3F}	\>   {\arabicfont\char"FD3F} 	\>	¦((¦ \\
% \textsf{FDF0}	\>   {\arialuni\char"FDF0} 	\>	¦SALLASTOP¦ \\
% \textsf{FDF1}	\>   {\arialuni\char"FDF1} 	\>	¦QALA¦ \\
% \textsf{FDF2}	\>   {\arabicfont\char"FDF2}   	\>	¦al-ll_ah¦ or ¦ALLAH¦ ** \\
% \textsf{FDF3}	\>   {\arialuni\char"FDF3} 	\>	¦AKBAR¦ \\
% \textsf{FDF4}	\>   {\arialuni\char"FDF4} 	\>	¦MUHAMMAD¦ \\
% \textsf{FDF5}	\>   {\arialuni\char"FDF5} 	\>	¦SALAM¦ \\
% \textsf{FDF6}	\>   {\arialuni\char"FDF6} 	\>	¦RASUL¦ or ¦RASOUL¦ \\
% \textsf{FDF7}	\>   {\arialuni\char"FDF7} 	\>	¦ALAYHI¦ or ¦ALAYHE¦ \\
% \textsf{FDF8}	\>   {\arialuni\char"FDF8} 	\>	¦WASALLAM¦ \\
% \textsf{FDF9}	\>   {\arialuni\char"FDF9} 	\>	¦SALLA¦ \\
% \textsf{FDFA}	\>   {\adobearabic\char"FDFA} 	\>	¦SLM¦ \\
% \textsf{FDFB}	\>   {\arialuni\char"FDFB} 	\>	¦JALLA¦ \\
% \textsf{FDFC}	\>   {\arabicfont\char"FDFC} 	\>	¦RIYAL¦ \\
% \textsf{FDFD}	\>   {\Huge\arabesque\char"F050} \>	¦BASMALA¦ \\ 
% \textbf{Non-Unicode Ligatures} \\
% 		 \>  {\tradarabic\char"FDF2}	\>	¦ll_ah¦ or ¦LLAH¦ ** \\
% 		 \>  {\adobearabic فَلِله} 	\> 	¦FALILLAH¦ ***
% \end{tabbing}
% \end{multicols}
% \parindentoff
% 
% \baselineskip 14pt 
% \textit{Notes}\\
% * Since most of these glyphs are not present in Scheherazade 
% (and, with a few exceptions, are very rarely featured in other fonts), 
% for illustration purposes we have taken \textsf{FDF0}, \textsf{FDF1}, 
% \textsf{FDF3}--\textsf{FDF9} and \textsf{FDFB} from Arial Unicode MS, 
% \textsf{FDFA} from Adobe Arabic, and \textsf{FDFD} from AGA Arabesque.
% 
% ** See §~\ref{allahliga} 
% 
% *** Specific to the font Adobe Arabic
% 
% \newpage
% \appendix
% \small
% \section{Notes on available free and commercial fonts for the Perso-Arabic script} 
% \label{arabicfont}
% \texttt{TO BE COMPLETED}
% 
% 
% ^^AThere are few free good quality fonts that support the Arabic script.
% 
% \begin{compactitem}[\textbf{·}]
% \item ‘Scheherazade’ and ‘Lateef’ from SIL
% 
% \item ‘Adobe Arabic’
% 
% \item Free fonts from \href{http://www.arabeyes.org}{Arabeyes.org}
% 
% \item \ldots
% 
% ^^AAdobe Arabic, dlig: (discretionary ligatures)
% ^^A
% ^^Aﻓ  َ ﻠ  ِ ﻠ ﻪ falilh > uni0641064E0644065006440647
% ^^A
% ^^Aﻟ ﻠ ﻪ l-l-h > uni064406440647
% ^^A
% ^^AA-l-l-h > FDF2
% 
% \item Arabic fonts on Mac~OS~X: ‘Geeza Pro’, ‘DecoType Naskh’, \ldots more?\\
% 	\textcolor{red}{$\rightarrow$ \texttt{Help needed to test those fonts!}}
% 
% \item ‘Arabic Typesetting’ and other Arabic fonts licensed to Microsoft
% 
% \item Commercial fonts by vendors: 
% 	\begin{compactitem}[\textbf{--}]
% 		\item Linotype GmbH: \ldots
% 		\item AGFA Monotype: \ldots
% 		\item ParaType: \ldots
% 		\item \ldots
% 	\end{compactitem}
% \end{compactitem}
% 
% 
% 
% \section{Recommended Unicode fonts for transliterating Oriental languages} 
% \label{latinextfonts}
% 
% Here follows a list of open source and freeware fonts with full 
% \textsc{Latin Extended Additional} coverage.\footnote{ ^^A
% 	See this \href{http://www.alanwood.net/unicode/fonts.html\#general}{webpage} for more details.}
% \begin{compactitem}[\textbf{·}]
% \item  Gentium, Doulos, Charis:  \url{http://scripts.sil.org/FontDownloads}
% 
% \item  Junicode: \url{http://junicode.sourceforge.net/}
% 
% \item  Libertine: \url{http://linuxlibertine.sourceforge.net/}
% 
% \item  DejaVu Serif / Sans / Sans Mono: \url{http://dejavu.sourceforge.net/wiki/index.php/Main_Page}
% 
% \item  Lucida Grande: Mac~OS~X
% 
% \item  TITUS Cyberbit Basic: \url{http://titus.fkidg1.uni-frankfurt.de/unicode/tituut.asp}
% 
% \item  Thryomanes: \url{ftp://ftp.io.com/pub/usr/hmiller/fonts/Thryomanes12.zip}
% 
% \item  jGaramond: \url{http://www.janthor.de/jGaramond/}
% 
% \item  Everson Mono Unicode (shareware): \url{http://www.evertype.com/emono/}
% 
% \item  Arial Unicode MS: Windows 
% 
% \item  Microsoft Sans Serif: Windows
% 
% \item  FreeSerif / FreeSans / FreeMono: \url{http://savannah.nongnu.org/projects/freefont/}
% 
% \item  Roman Unicode: \url{http://everywitchway.net/linguistics/fonts/roman.html}
% 
% \item  Chrisanthi Unicode: \url{http://everywitchway.net/linguistics/fonts/chrysuni.html}
% 
% \item  HindSight Unicode (not very nice): \url{http://dartcanada.tripod.com/Objets/Zips/HindUnic.zip}
% 
% \item  AbRoman: \url{http://www.languagegeek.com/font/fontdownload.html}
% 
% \item  Garava: \url{http://www.aimwell.org/Fonts/fonts.html}
% 
% \item  Verajja (based on Bitstream Vera Sans, so very similar to DejaVu Sans): 
% 	\url{http://www.aimwell.org/Fonts/fonts.html}
% 
% \item  Legendum: \url{http://home.kabelfoon.nl/~slam/fonts/fonts.html}
% 
% \item  Code2000 (shareware): \url{http://www.code2000.net}
% \end{compactitem}
% 
% 
% \section{Implementation}
%\iffalse
%<*package>
%\fi
%    \begin{macrocode}
\NeedsTeXFormat{LaTeX2e}
\ProvidesPackage{arabxetex}
  [2010/01/08 v1.1.4 ArabTeX-like interface for XeLaTeX]
%
\DeclareOption{fullvoc}{\def\ax@mode{fullvoc}}
\DeclareOption{voc}{\def\ax@mode{voc}}
\DeclareOption{novoc}{\def\ax@mode{novoc}}
\DeclareOption{trans}{\def\ax@mode{trans}}
\DeclareOption{utf}{\def\ax@mode{utf}}
\DeclareOption{fdf2alif}{\def\ax@font@allah{fdf2alif}}
\DeclareOption{fdf2noalif}{\def\ax@font@allah{fdf2noalif}}
\newif\ifmirror@punct\mirror@punctfalse
\DeclareOption{mirrorpunct}{\mirror@puncttrue}
\ExecuteOptions{novoc,fdf2alif}
\ProcessOptions
\def\ax@mode@fullvoc{fullvoc}
\def\ax@mode@voc{voc}
\def\ax@mode@novoc{novoc}
\def\ax@mode@trans{trans}
\def\ax@mode@utf{utf}
\newif\ifax@mode@defined
\def\ax@ismode@defined#1{%
	\ifcsname ax@mode@#1\endcsname%
		\ax@mode@definedtrue%
	\else%
		\ax@mode@definedfalse%
	\fi}
\def\ax@lang{arab}%default language (for macros like \aemph)
\RequirePackage{amsmath}%because of macro \overline used in \aemph
\RequirePackage{fontspec}
\RequirePackage{bidi}
\AtBeginDocument{\ifdefined\arabicfont\relax\else%
	\PackageWarning{arabxetex}{\string\arabicfont\ is not defined!^^JI will try to load Scheherazade}%
	\newfontfamily\arabicfont[Script=Arabic,Scale=2]{Scheherazade}%
	\let\tmp@baselineskip=\baselineskip%
	\baselineskip=2*\tmp@baselineskip%
	\fi%
	}%
\def\ax@trans@style{\itshape}%
\newcommand{\SetTranslitStyle}[1]{\def\ax@trans@style{#1}}
\newcommand{\SetTranslitConvention}[1]{\def\ax@trans@convention{#1}}
\def\ax@trans@convention{loc}% Library of Congress is default
\newcommand{\SetAllahWithAlif}{\def\ax@font@allah{fdf2alif}}
\newcommand{\SetAllahWithoutAlif}{\def\ax@font@allah{fdf2noalif}}
\def\utf@fontfeature{\ifmirror@punct\addfontfeature{Mapping=mirrorpunct}\else\relax\fi}
\def\arabtex@codes{\catcode`^=11\relax\catcode`_=11\relax}
\def\UC{\char"E000} % This is used to capitalize the following letter (ignoring
		    % article al-) in transliteration mode
%%%Font setup
\def\ax@farsi@font{\ifdefined\farsifont\farsifont\else\arabicfont\fi}
\def\ax@urdu@font{\ifdefined\urdufont\urdufont\else\arabicfont\fi}
\def\ax@pashto@font{\ifdefined\pashtofont\pashtofont\else\arabicfont\fi}
\def\ax@maghribi@font{\ifdefined\maghribifont\maghribifont\else\arabicfont\fi}
\def\ax@sindhi@font{\ifdefined\sindhifont\sindhifont\else\arabicfont\fi}
\def\ax@kashmiri@font{\ifdefined\kashmirifont\kashmirifont\else\arabicfont\fi}
\def\ax@ottoman@font{\ifdefined\ottomanfont\ottomanfont\else\arabicfont\fi}
\def\ax@kurdish@font{\ifdefined\kurdishfont\kurdishfont\else\arabicfont\fi}
\def\ax@malay@font{\ifdefined\malayfont\malayfont\else\arabicfont\fi}
\def\ax@uighur@font{\ifdefined\uighurfont\uighurfont\else\arabicfont\fi}
\def\ax@urdu@font{\ifdefined\urdufont\urdufont\else\arabicfont\fi}

\newenvironment{arab}[1][\ax@mode]%
{\edef\@tempa{#1}%
\def\ax@lang{arab}%
\ax@ismode@defined{\@tempa}%
\ifax@mode@defined%
	\ifx\@tempa\ax@mode@trans%
		\par\arabtex@codes\ax@trans@style%
		\addfontfeature{Mapping=arabtex-trans-\ax@trans@convention}%
	\else%
	\ifx\@tempa\ax@mode@utf%
		\par\setRL\arabicfont\utf@fontfeature%
	\else%
		\par\setRL\arabicfont\arabtex@codes%
		\addfontfeature{Mapping=arabtex-\ax@font@allah-\@tempa}%
	\fi\fi%
\else%
	\PackageWarning{arabxetex}{Mode \@tempa\ not defined, defaulting to \@ax@mode}%
	\par\setRL\arabicfont\arabtex@codes%
	\addfontfeature{Mapping=arabtex-\ax@font@allah-\ax@mode}%
\fi}
{\ifx\@tempa\ax@mode@trans\relax\else\unsetRL\fi\par}
\let\Arabic=\arab%
%%%
\newenvironment{maghribi}[1][\ax@mode]%
{\edef\@tempa{#1}%
\def\ax@lang{maghribi}%
\ax@ismode@defined{\@tempa}%
\ifax@mode@defined%
	\ifx\@tempa\ax@mode@trans%
		\par\arabtex@codes\ax@trans@style%
		\addfontfeature{Mapping=arabtex-trans-\ax@trans@convention}%
	\else%
	\ifx\@tempa\ax@mode@utf%
		\par\setRL\ax@maghribi@font\utf@fontfeature%
	\else%
		\par\setRL\ax@maghribi@font\arabtex@codes%
		\addfontfeature{Mapping=arabtex-\ax@font@allah-maghribi-\@tempa}%
	\fi\fi%
\else%
	\PackageWarning{arabxetex}{Mode \@tempa\ not defined, defaulting to \@ax@mode}%
	\par\setRL\ax@maghribi@font\arabtex@codes%
	\addfontfeature{Mapping=arabtex-\ax@font@allah-maghribi-\ax@mode}%
\fi}
{\ifx\@tempa\ax@mode@trans\relax\else\unsetRL\fi\par}
%%%
\newenvironment{farsi}[1][\ax@mode]%
{\edef\@tempa{#1}%
\def\ax@lang{farsi}%
\ax@ismode@defined{\@tempa}%
\ifax@mode@defined%
	\ifx\@tempa\ax@mode@trans%
		\par\arabtex@codes\ax@trans@style%
		\addfontfeature{Mapping=arabtex-farsi-trans-\ax@trans@convention}%
	\else%
	\ifx\@tempa\ax@mode@utf%
		\par\setRL\ax@farsi@font\utf@fontfeature%
	\else%
		\par\setRL\ax@farsi@font\arabtex@codes%
		\addfontfeature{Mapping=arabtex-\ax@font@allah-farsi-\@tempa}%
	\fi\fi%
\else%
	\PackageWarning{arabxetex}{Mode \@tempa\ not defined, defaulting to \@ax@mode}%
	\par\setRL\ax@farsi@font\arabtex@codes%
	\addfontfeature{Mapping=arabtex-\ax@font@allah-farsi-\ax@mode}%
\fi}
{\ifx\@tempa\ax@mode@trans\relax\else\unsetRL\fi\par}
\let\persian=\farsi
%%%
\newenvironment{urdu}[1][\ax@mode]%
{\edef\@tempa{#1}%
\def\ax@lang{urdu}%
\ax@ismode@defined{\@tempa}%
\ifax@mode@defined%
	\ifx\@tempa\ax@mode@trans%
		\par\arabtex@codes\ax@trans@style%
		\addfontfeature{Mapping=arabtex-urdu-trans-\ax@trans@convention}%
	\else%
	\ifx\@tempa\ax@mode@utf%
		\par\setRL\ax@urdu@font\addfontfeature{Language=Urdu}%Mapping=arabtex-utf}%
	\else%
		\par\setRL\ax@urdu@font\arabtex@codes%
		\addfontfeature{Mapping=arabtex-\ax@font@allah-urdu-\@tempa}%
	\fi\fi%
\else%
	\PackageWarning{arabxetex}{Mode \@tempa\ not defined, defaulting to \@ax@mode}%
	\par\setRL\ax@urdu@font\arabtex@codes%
	\addfontfeature{Mapping=arabtex-\ax@font@allah-urdu-\ax@mode}%
\fi}
{\ifx\@tempa\ax@mode@trans\relax\else\unsetRL\fi\par}
%%%
\newenvironment{pashto}[1][\ax@mode]%
{\edef\@tempa{#1}%
\def\ax@lang{pashto}%
\ax@ismode@defined{\@tempa}%
\ifax@mode@defined%
	\ifx\@tempa\ax@mode@trans%
		\par\arabtex@codes\ax@trans@style%
		\addfontfeature{Mapping=arabtex-pashto-trans-\ax@trans@convention}%
	\else%
	\ifx\@tempa\ax@mode@utf%
		\par\setRL\ax@pashto@font\utf@fontfeature%
	\else%
		\par\setRL\ax@pashto@font\arabtex@codes%
		\addfontfeature{Mapping=arabtex-\ax@font@allah-pashto-\@tempa}%
	\fi\fi%
\else%
	\PackageWarning{arabxetex}{Mode \@tempa\ not defined, defaulting to \@ax@mode}%
	\par\setRL\ax@pashto@font\arabtex@codes%
	\addfontfeature{Mapping=arabtex-\ax@font@allah-pashto-\ax@mode}%
\fi}
{\ifx\@tempa\ax@mode@trans\relax\else\unsetRL\fi\par}
%%%
\newenvironment{sindhi}[1][\ax@mode]%
{\edef\@tempa{#1}%
\def\ax@lang{sindhi}%
\ax@ismode@defined{\@tempa}%
\ifax@mode@defined%
	\ifx\@tempa\ax@mode@trans%
		\par\arabtex@codes\ax@trans@style%
		\addfontfeature{Mapping=arabtex-sindhi-trans-\ax@trans@convention}%
	\else%
	\ifx\@tempa\ax@mode@utf%
		\par\setRL\ax@sindhi@font\addfontfeature{Language=Sindhi}%Mapping=arabtex-utf}%
	\else%
		\par\setRL\ax@sindhi@font\arabtex@codes%
		\addfontfeature{Mapping=arabtex-\ax@font@allah-sindhi-\@tempa,Language=Sindhi}%
	\fi\fi%
\else%
	\PackageWarning{arabxetex}{Mode \@tempa\ not defined, defaulting to \@ax@mode}%
	\par\setRL\ax@sindhi@font\arabtex@codes%
	\addfontfeature{Mapping=arabtex-\ax@font@allah-sindhi-\ax@mode,Language=Sindhi}%
\fi}
{\ifx\@tempa\ax@mode@trans\relax\else\unsetRL\fi\par}
%%%
\newenvironment{ottoman}[1][\ax@mode]%
{\edef\@tempa{#1}%
\def\ax@lang{ottoman}%
\ax@ismode@defined{\@tempa}%
\ifax@mode@defined%
	%\ifx\@tempa\ax@mode@trans%
	%	\par\arabtex@codes\ax@trans@style%
	%	\addfontfeature{Mapping=arabtex-turk-trans-\ax@trans@convention}%
	%\else%
	\ifx\@tempa\ax@mode@utf%
		\par\setRL\ax@ottoman@font\utf@fontfeature%
	\else%
		\par\setRL\ax@ottoman@font\arabtex@codes%
		\addfontfeature{Mapping=arabtex-\ax@font@allah-turk-\@tempa}%
	\fi%\fi%
\else%
	\PackageWarning{arabxetex}{Mode \@tempa\ not defined, defaulting to \@ax@mode}%
	\par\setRL\ax@ottoman@font\arabtex@codes%
	\addfontfeature{Mapping=arabtex-\ax@font@allah-turk-\ax@mode}%
\fi}
{%\ifx\@tempa\ax@mode@trans\relax\else
\unsetRL%\fi
\par}
\let\turk=\ottoman
%%%
\newenvironment{kurdish}[1][]%
{\def\ax@lang{kurdish}%
%\edef\@tempa{#1}%
%\ifx\@tempa\ax@mode@trans%
%	\par\arabtex@codes\ax@trans@style%
%	\addfontfeature{Mapping=arabtex-kurdish-trans-\ax@trans@convention}%
%\else%
\ifx\@tempa\ax@mode@utf%
	\par\setRL\ax@kurdish@font\addfontfeature{Language=Kurdish}%Mapping=arabtex-utf}%
\else%
	\par\setRL\ax@kurdish@font\arabtex@codes%
	\addfontfeature{Mapping=arabtex-\ax@font@allah-kurdish,Language=Kurdish}%
\fi}
{%\ifx\@tempa\ax@mode@trans\relax\else
\unsetRL%\fi
\par}
%%%
\newenvironment{kashmiri}[1][\ax@mode]%
{\edef\@tempa{#1}%
\def\ax@lang{kashmiri}%
\ax@ismode@defined{\@tempa}%
\ifax@mode@defined%
	%\ifx\@tempa\ax@mode@trans%
	%	\par\arabtex@codes\ax@trans@style%
	%	\addfontfeature{Mapping=arabtex-\ax@font@allah-kashmiri-trans-\ax@trans@convention}%
	%\else%
	\ifx\@tempa\ax@mode@utf%
		\par\setRL\ax@kashmiri@font\utf@fontfeature%
	\else%
		\par\setRL\ax@kashmiri@font\arabtex@codes%
		\addfontfeature{Mapping=arabtex-\ax@font@allah-kashmiri-\@tempa}%
	\fi%\fi%
\else%
	\PackageWarning{arabxetex}{Mode \@tempa\ not defined, defaulting to \@ax@mode}%
	\par\setRL\ax@kashmiri@font\arabtex@codes%
	\addfontfeature{Mapping=arabtex-\ax@font@allah-kashmiri-\ax@mode}%
\fi}
{%\ifx\@tempa\ax@mode@trans\relax\else
\unsetRL%\fi
\par}
%%%
\newenvironment{malay}[1][\ax@mode]%
{\edef\@tempa{#1}%
\def\ax@lang{malay}%
\ax@ismode@defined{\@tempa}%
\ifax@mode@defined%
	%\ifx\@tempa\ax@mode@trans%
	%	\par\arabtex@codes\ax@trans@style%
	%	\addfontfeature{Mapping=arabtex-malay-trans-\ax@trans@convention}%
	%\else%
	\ifx\@tempa\ax@mode@utf%
		\par\setRL\ax@malay@font\utf@fontfeature%
	\else%
		\par\setRL\ax@malay@font\arabtex@codes%
		\addfontfeature{Mapping=arabtex-\ax@font@allah-malay-\@tempa}%
	\fi%\fi%
\else%
	\PackageWarning{arabxetex}{Mode \@tempa\ not defined, defaulting to \@ax@mode}%
	\par\setRL\ax@malay@font\arabtex@codes%
	\addfontfeature{Mapping=arabtex-\ax@font@allah-malay-\ax@mode}%
\fi}
{%\ifx\@tempa\ax@mode@trans\relax\else
\unsetRL%\fi
\par}
\let\jawi=\malay
%%%
\newenvironment{uighur}[1]%
{%\edef\@tempa{#1}%
\def\ax@lang{uighur}%
%\ifx\@tempa\ax@mode@trans%
%	\par\arabtex@codes\ax@trans@style%
%	\addfontfeature{Mapping=arabtex-uighur-trans-\ax@trans@convention}%
%\else%
\ifx\@tempa\ax@mode@utf%
	\par\setRL\ax@uighur@font\utf@fontfeature%
\else%
	\par\setRL\ax@uighur@font\arabtex@codes%
	\addfontfeature{Mapping=arabtex-\ax@font@allah-uighur}%
\fi}
{%\ifx\@tempa\ax@mode@trans\relax\else
\unsetRL%\fi
\par}
%%%
\def\textarab{\bgroup\arabtex@codes\text@arab}
\let\textarabic=\textarab
\def\textmaghribi{\bgroup\arabtex@codes\text@maghribi}
\def\textfarsi{\bgroup\arabtex@codes\text@farsi}
\let\textpersian=\textfarsi
\def\texturdu{\bgroup\arabtex@codes\text@urdu}
\def\textsindhi{\bgroup\arabtex@codes\text@sindhi}
\def\textpashto{\bgroup\arabtex@codes\text@pashto}
\def\textottoman{\bgroup\arabtex@codes\text@ottoman}
\let\textturk=\textottoman
\def\textkurdish{\bgroup\arabtex@codes\text@kurdish}
\def\textkashmiri{\bgroup\arabtex@codes\text@kashmiri}
\def\textmalay{\bgroup\arabtex@codes\text@malay}
\let\textjawi=\textmalay
\def\textuighur{\bgroup\arabtex@codes\text@uighur}
\newcommand\text@arab[2][\ax@mode]{%
	\edef\@tempa{#1}%
	\def\ax@lang{arab}%
	\ax@ismode@defined{\@tempa}%
	\ifax@mode@defined%
		\ifx\@tempa\ax@mode@trans%
			{\ax@trans@style\addfontfeature{Mapping=arabtex-trans-\ax@trans@convention}#2}%
		\else%
		\ifx\@tempa\ax@mode@utf%
			\RL{\arabicfont\utf@fontfeature #2}%
		\else%
			\RL{\arabicfont\addfontfeature{Mapping=arabtex-\ax@font@allah-\@tempa}#2}%
		\fi\fi%
	\else%
	\PackageWarning{arabxetex}{Mode \@tempa\ not defined, defaulting to \@ax@mode}%
	\RL{\arabicfont\addfontfeature{Mapping=arabtex-\ax@font@allah-\ax@mode}#2}%
	\fi\egroup}
\newcommand\text@maghribi[2][\ax@mode]{%
	\edef\@tempa{#1}%
	\def\ax@lang{maghribi}%
	\ax@ismode@defined{\@tempa}%
	\ifax@mode@defined%
		\ifx\@tempa\ax@mode@trans%
			{\ax@trans@style\addfontfeature{Mapping=arabtex-trans-\ax@trans@convention}#2}%
		\else%
		\ifx\@tempa\ax@mode@utf%
			\RL{\ax@maghribi@font\utf@fontfeature #2}%
		\else%
			\RL{\ax@maghribi@font%
			\addfontfeature{Mapping=arabtex-\ax@font@allah-maghribi-\@tempa}#2}%
		\fi\fi%
	\else%
	\PackageWarning{arabxetex}{Mode \@tempa\ not defined, defaulting to \@ax@mode}%
	\RL{\ax@maghribi@font\addfontfeature{Mapping=arabtex-\ax@font@allah-maghribi-\ax@mode}#2}%
	\fi\egroup}
\newcommand\text@farsi[2][\ax@mode]{%
	\edef\@tempa{#1}%
	\def\ax@lang{farsi}%
	\ax@ismode@defined{\@tempa}%
	\ifax@mode@defined%
		\ifx\@tempa\ax@mode@trans%
			{\ax@trans@style\addfontfeature{Mapping=arabtex-farsi-trans-\ax@trans@convention}#2}%
		\else%
		\ifx\@tempa\ax@mode@utf%
			\RL{\ax@farsi@font\utf@fontfeature #2}%
		\else%
			\RL{\ax@farsi@font%
			\addfontfeature{Mapping=arabtex-\ax@font@allah-farsi-\@tempa}#2}%
		\fi\fi%
	\else%
	\PackageWarning{arabxetex}{Mode \@tempa\ not defined, defaulting to \@ax@mode}%
	\RL{\ax@farsi@font\addfontfeature{Mapping=arabtex-\ax@font@allah-farsi-\ax@mode}#2}%
	\fi\egroup}
\newcommand\text@urdu[2][\ax@mode]{%
	\edef\@tempa{#1}%
	\def\ax@lang{urdu}%
	\ax@ismode@defined{\@tempa}%
	\ifax@mode@defined%
		\ifx\@tempa\ax@mode@trans%
			{\ax@trans@style\addfontfeature{Mapping=arabtex-urdu-trans-\ax@trans@convention}#2}%
		\else%
		\ifx\@tempa\ax@mode@utf%
			\RL{\ax@urdu@font\addfontfeature{Language=Urdu}#2}%eventually Mapping=arabtex-utf
		\else%
			\RL{\ax@urdu@font%
			\addfontfeature{Mapping=arabtex-\ax@font@allah-urdu-\@tempa,Language=Urdu}#2}%
		\fi\fi%
	\else%
	\PackageWarning{arabxetex}{Mode \@tempa\ not defined, defaulting to \@ax@mode}%
	\RL{\ax@urdu@font\addfontfeature{Mapping=arabtex-\ax@font@allah-urdu-\ax@mode,Language=Urdu}#2}%
	\fi\egroup}
\newcommand\text@sindhi[2][\ax@mode]{%
	\edef\@tempa{#1}%
	\def\ax@lang{sindhi}%
	\ax@ismode@defined{\@tempa}%
	\ifax@mode@defined%
		\ifx\@tempa\ax@mode@trans%
			{\ax@trans@style\addfontfeature{Mapping=arabtex-sindhi-trans-\ax@trans@convention}#2}%
		\else%
		\ifx\@tempa\ax@mode@utf%
			\RL{\ax@sindhi@font\addfontfeature{Language=Sindhi} #2}%eventually Mapping=arabtex-utf
		\else%
			\RL{\ax@sindhi@font%
			\addfontfeature{Mapping=arabtex-\ax@font@allah-sindhi-\@tempa,Language=Sindhi}#2}%
		\fi\fi%
	\else%
	\PackageWarning{arabxetex}{Mode \@tempa\ not defined, defaulting to \@ax@mode}%
	\RL{\ax@sindhi@font\addfontfeature{Mapping=arabtex-\ax@font@allah-sindhi-\ax@mode,Language=Sindhi}#2}%
	\fi\egroup}
\newcommand\text@pashto[2][\ax@mode]{%
	\edef\@tempa{#1}%
	\def\ax@lang{pashto}%
	\ax@ismode@defined{\@tempa}%
	\ifax@mode@defined%
		\ifx\@tempa\ax@mode@trans%
			{\ax@trans@style\addfontfeature{Mapping=arabtex-pashto-trans-\ax@trans@convention}#2}%
		\else%
		\ifx\@tempa\ax@mode@utf%
			\RL{\ax@pashto@font\utf@fontfeature #2}%
		\else%
			\RL{\ax@pashto@font%
			\addfontfeature{Mapping=arabtex-\ax@font@allah-pashto-\@tempa}#2}%
		\fi\fi%
	\else%
	\PackageWarning{arabxetex}{Mode \@tempa\ not defined, defaulting to \@ax@mode}%
	\RL{\ax@pashto@font\addfontfeature{Mapping=arabtex-\ax@font@allah-pashto-\ax@mode}#2}%
	\fi\egroup}
\newcommand\text@ottoman[2][\ax@mode]{%
	\edef\@tempa{#1}%
	\def\ax@lang{ottoman}%
	\ax@ismode@defined{\@tempa}%
	\ifax@mode@defined%
		% UNCOMMENT when transliteration mapping is done
		%\ifx\@tempa\ax@mode@trans%
		%	{\ax@trans@style\addfontfeature{Mapping=arabtex-turk-trans-\ax@trans@convention}#2}%
		%\else%
		\ifx\@tempa\ax@mode@utf%
			\RL{\ax@ottoman@font\utf@fontfeature #2}%
		\else%
			\RL{\ax@ottoman@font%
			\addfontfeature{Mapping=arabtex-\ax@font@allah-turk-\@tempa}#2}%
		\fi%\fi%
	\else%
	\PackageWarning{arabxetex}{Mode \@tempa\ not defined, defaulting to \@ax@mode}%
	\RL{\ax@ottoman@font\addfontfeature{Mapping=arabtex-\ax@font@allah-turk-\ax@mode}#2}%
	\fi\egroup}
\newcommand\text@kurdish[2][]{%
	\edef\@tempa{#1}%
	\def\ax@lang{kurdish}%
		% UNCOMMENT when transliteration mapping is done
		%\ifx\@tempa\ax@mode@trans%
		%	{\ax@trans@style\addfontfeature{Mapping=arabtex-kurdish-trans-\ax@trans@convention}#2}%
		%\else%
		\ifx\@tempa\ax@mode@utf%
			\RL{\ax@kurdish@font\utf@fontfeature #2}%
		\else%
			\RL{\ax@kurdish@font%
			\addfontfeature{Mapping=arabtex-\ax@font@allah-kurdish,Language=Kurdish}#2}%
		\fi%\fi%
	\egroup}
\newcommand\text@kashmiri[2][\ax@mode]{%
	\edef\@tempa{#1}%
	\def\ax@lang{kashmiri}%
	\ax@ismode@defined{\@tempa}%
	\ifax@mode@defined%
		% UNCOMMENT when transliteration mapping is done
		%\ifx\@tempa\ax@mode@trans%
		%	{\ax@trans@style\addfontfeature{Mapping=arabtex-kashmiri-trans-\ax@trans@convention}#2}%
		%\else%
		\ifx\@tempa\ax@mode@utf%
			\RL{\ax@kashmiri@font\utf@fontfeature #2}%
		\else%
			\RL{\ax@kashmiri@font%
			\addfontfeature{Mapping=arabtex-\ax@font@allah-kashmiri-\@tempa}#2}%
		\fi%\fi%
	\else%
	\PackageWarning{arabxetex}{Mode \@tempa\ not defined, defaulting to \@ax@mode}%
	\RL{\ax@kashmiri@font\addfontfeature{Mapping=arabtex-\ax@font@allah-kashmiri-\ax@mode}#2}%
	\fi\egroup}
\newcommand\text@malay[2][\ax@mode]{%
	\edef\@tempa{#1}%
	\def\ax@lang{malay}%
	\ax@ismode@defined{\@tempa}%
	\ifax@mode@defined%
		% UNCOMMENT when transliteration mapping is done
		%\ifx\@tempa\ax@mode@trans%
		%	{\ax@trans@style\addfontfeature{Mapping=arabtex-malay-trans-\ax@trans@convention}#2}%
		%\else%
		\ifx\@tempa\ax@mode@utf%
			\RL{\ax@malay@font\utf@fontfeature #2}%
		\else%
			\RL{\ax@malay@font%
			\addfontfeature{Mapping=arabtex-\ax@font@allah-malay-\@tempa}#2}%
		\fi%\fi%
	\else%
	\PackageWarning{arabxetex}{Mode \@tempa\ not defined, defaulting to \ax@mode}%
	\RL{\ax@malay@font\addfontfeature{Mapping=arabtex-\ax@font@allah-malay-\ax@mode}#2}%
	\fi\egroup}
\newcommand\text@uighur[2][]{%
	\edef\@tempa{#1}%
	\def\ax@lang{uighur}%
		% UNCOMMENT when transliteration mapping is done
		%\ifx\@tempa\ax@mode@trans%
		%	{\ax@trans@style\addfontfeature{Mapping=arabtex-uighur-trans-\ax@trans@convention}#2}%
		%\else%
		\ifx\@tempa\ax@mode@utf%
			\RL{\ax@uighur@font\utf@fontfeature #2}%
		\else%
			\RL{\ax@uighur@font%
			\addfontfeature{Mapping=arabtex-\ax@font@allah-uighur}#2}%
		\fi%\fi%
	\egroup}
%
\newcommand{\textLR}[1]{\LR{\rmfamily #1}}
\newcommand{\aemph}[1]{%
  \edef\@tempb{\expandafter\noexpand\csname text\ax@lang\endcsname}%
  $\overline{\text{\@tempb{#1}}}$}
%
% experimental: this surely does not work!
% I want a mechanism to enable defining commands that accept args containing \arabtex@codes
%\def\newarabxetexcommand#1#2{%
%	\edef\ax@tcomm@nd{\expandafter\relax\csname #1@\endcsname}
%	\expandafter\def\string#1{%
%	\bgroup\arabtex@codes\ax@tcomm@nd}
%	\def\ax@tcomm@nd{#2\egroup}}
%    \end{macrocode}
%
%\iffalse
%</package>
%\fi
% 
% \subsection{Some notes on the TECkit mappings}
% 
% \texttt{TO BE DONE \dots}
% 
% 
% ^^A\section{Unicode Matters} ---> TODO CONTACT ROOZBEH ABOUT THIS
% ^^A
% ^^ASome glyphs are missing in Unicode:
% ^^A
% ^^A- wavy Hamza below as combining diacritic (for Kashmiri)
% ^^A
% ^^A- .b with ring below (for Kashmiri) (NB: can be poorly rendered by U+066E U+06EA)
% ^^A
% ^^A- hamza directly above baseline (for Qur'an) : 0654 is too high!
% ^^A  \def\hamzaB{\raisebox[-1ex]{\char"0654} is ok
% ^^A
% ^^A- U+FDF2: codify the LLAH ligature as well?
% ^^A          Perhaps the FALILLAH ligature could be added as well?
%
% ^^A\clearpage
% ^^A\PrintChanges
%
% ^^A\clearpage
% ^^A\PrintIndex
%
% \Finale
%
%\iffalse
%<*dtx-style>
%    \begin{macrocode}
\ProvidesPackage{arabxetex-dtx-style}
\usepackage{fontspec,xltxtra}
\usepackage{array,supertabular,xspace,multicol,fancyvrb,paralist}
\usepackage[voc]{arabxetex}
\usepackage[dvipdfm]{color}
\definecolor{myblue}{rgb}{0.02,0.04,0.48}
\definecolor{myred}{rgb}{0.65,0.04,0.07}
\definecolor{darkgray}{gray}{0.3}
\definecolor{lightgray}{gray}{0.6}
\definecolor{IslamicGreen}{rgb}{0,.43,0}
\usepackage[dvipdfm,
    bookmarks=true,
    colorlinks=true,
    linkcolor=myblue,
    urlcolor=myblue,
    citecolor=myblue,
    hyperindex=false,
    hyperfootnotes=false,
    pdftitle={The ArabXeTeX package},
    pdfauthor={F Charette <firmicus@gmx.net>},
    pdfsubject={An ArabTeX-like interface for typesetting languages in Arabic script with XeLaTeX},
    pdfkeywords={ArabTeX, XeTeX, XeLaTeX, Arabic, Maghribi, Persian, Farsi, Urdu, Sindhi, 
    	Pashto, Turkish, Ottoman, Kurdish, Kashmiri, Malay, Jawi, Uighur}
    ]{hyperref}
% NB some of this preamble is taken or adapted from fontspec-doc-style.sty
\newcommand*\pkg[1]{\textsf{#1}}
\def\eg{\textit{e.g.,}\xspace}
\def\ie{\textit{i.e.,}\xspace}
\def\ca{\textit{ca.}\@\xspace}
\def\Eg{\textit{E.g.,}\xspace}
\def\Ie{\textit{I.e.,}\xspace}
\def\etc{\@ifnextchar.{\textit{etc}}{\textit{etc.}\@\xspace}}

\newlength{\oldparindent}
\newcommand{\parindentoff}{\setlength{\oldparindent}{\parindent} \setlength{\parindent}{0pt}}
\newcommand{\parindenton}{\setlength{\parindent}{\oldparindent}}

%% LOGOS, tuned for Palatino:
\makeatletter
\TeX@logo@spacing{-0.12em}{-0.12em}%
  {0.5ex}{-0.3em}{-0.12em}{-0.1em}
\makeatother
\def\arabxetex{Arab\XeTeX}
\def\arabtex{Arab\TeX}
\def\arabxetexAr{\RL{\fontspec[Script=Arabic,Scale=1.5]{DecoType Thuluth} عرب زيتخ}}
\def\MacOSX{Mac~OS~X}
\def\ASCII{\textsc{ascii}} % ii small caps does not work with Junicode!

%% Sidenotes  << copied from fontspec.dtx
\newcommand\warn[2]{%
  \edef\thisversion{#1}%
  \ifhmode\unskip~\fi{\ifx\thisversion\fileversion\color{red}\fi
  $\leftarrow$}%
  \marginpar{\raggedleft
    \small%
      {\ifx\thisversion\fileversion\color{red}\fi$\rightarrow$}%
      \,{\notefont #1:~#2}}}

\newcommand\new[1]{%
  \edef\thisversion{#1}%
  \ifhmode\unskip~\fi{\ifx\thisversion\fileversion\color{blue}\else\color[gray]{0.5}\fi
  $\leftarrow$}%
  \marginpar{\raggedleft
    \small\ifx\thisversion\fileversion\color{blue}\else\color[gray]{0.5}\fi
    $\rightarrow$\,{\notefont #1:~New!}}}

%% Sidenote font
\newfontfamily\notefont[Scale=MatchLowercase]{Lucida Sans Unicode}

%%%

%% fontspec declarations:
\setromanfont[Mapping=tex-text]{Junicode}%{FPL Neu}
\defaultfontfeatures{Scale=MatchLowercase}
\setmonofont{Inconsolata}%{Latin Modern Typewriter Proportional}%{Luxi Mono}%{Courier New}
\setsansfont{Lucida Sans Unicode}
%\setmathrm{Optima Regular}
%\setboldmathrm[BoldFont=Optima ExtraBlack]{Optima Bold}
%\defaultfontfeatures{Mapping=tex-text}
\newfontfamily\arabicfont[Script=Arabic,Scale=1.5]{Scheherazade}
\newfontfamily\urdufont[Script=Arabic,Scale=1.2]{Nafees Pakistani Naskh}%Nafees Nastaleeq does not work well
\newfontfamily\farsifont[Script=Arabic,Scale=1.33]{Farsi Simple Bold}
\newfontfamily\lateeffont[Script=Arabic,Scale=1.5]{Lateef}
\newfontfamily\lotusfont[Script=Arabic,Scale=1.5]{Lotus Linotype}
\newfontfamily\arabtype[Script=Arabic,Scale=1.5]{Arabic Typesetting}
\newfontfamily\adobearabic[Script=Arabic,Scale=1.5]{Adobe Arabic}
\newfontfamily\tradarabic[Script=Arabic,Scale=1.5]{Traditional Arabic}
\newfontfamily\arabesque{AGA Arabesque}
\newfontfamily\arialuni{Arial Unicode MS}
\newfontfamily\gentium{Gentium}

\linespread{1.05}
\frenchspacing
%\newenvironment{MYitemize}{%
%\renewcommand\labelitemi{\cdot}%
%\begin{list}{}{\setlength{\itemsep}{.15\parsep}}}%
%%\setlength{\labelwidth}{0pt}\setlength{\leftmargin}{0pt}\setlength{\itemindent}{0pt}}
%{\end{list}}
%\addtolength{\textheight}{3cm}
%\setlength{\leftmargin}{1cm}
%\setlength{\topmargin}{0cm}
%\newcommand{\Tarab}[2][black]{\textcolor{#1}{\textarab{ #2 }}}
%\newcommand{\Tfarsi}[2][black]{\textcolor{#1}{\textfarsi{ #2 }}}
%\newcommand{\Turdu}[2][black]{\textcolor{#1}{\texturdu{ #2 }}}
%\newcommand{\Tsindhi}[2][black]{\textcolor{#1}{\textsindhi{ #2 }}}
%\newcommand{\Tpashto}[2][black]{\textcolor{#1}{\textpashto{ #2 }}}
%\newcommand{\Tkashmiri}[2][black]{\textcolor{#1}{\textkashmiri{ #2 }}}
%\newcommand{\Tkurdish}[2][black]{\textcolor{#1}{\textkurdish{ #2 }}}
%\newcommand{\Tturk}[2][black]{\textcolor{#1}{\textturk{ #2 }}}
%\newcommand{\Tmalay}[2][black]{\textcolor{#1}{\textmalay{ #2 }}}
%\newcommand{\Tuighur}[2][black]{\textcolor{#1}{\textuighur{ #2 }}}
%%%
\def\GN{\bgroup\catcode`\_=11\relax\catcode`\^=11\relax\GNX}
\def\GR{\bgroup\catcode`\_=11\relax\catcode`\^=11\relax\GRX}
\def\GB{\bgroup\catcode`\_=11\relax\catcode`\^=11\relax\GBX}
\def\GG{\bgroup\catcode`\_=11\relax\catcode`\^=11\relax\GGX}
\def\GV{\bgroup\catcode`\_=11\relax\catcode`\^=11\relax\GVX}
\newcommand{\GNX}[3][arab]{%
	\edef\@tempa{\expandafter\noexpand\csname text#1\endcsname}
	\parbox[c]{7.5mm}{\centering{\@tempa{#2}}\\[1pt]\ttfamily\fontsize{5}{5}\selectfont{#3}}\egroup}
\newcommand{\GRX}[3][arab]{%
	\edef\@tempa{\expandafter\noexpand\csname text#1\endcsname}
	\parbox[c]{7.5mm}{\centering{\textcolor{myred}{\@tempa{#2}}}\\[1pt]\ttfamily\fontsize{5}{5}\selectfont{#3}}\egroup}
\newcommand{\GBX}[3][arab]{%
	\edef\@tempa{\expandafter\noexpand\csname text#1\endcsname}
	\parbox[c]{7.5mm}{\centering{\textcolor{myblue}{\@tempa{#2}}}\\[1pt]\ttfamily\fontsize{5}{5}\selectfont{#3}}\egroup}
\newcommand{\GVX}[3][arab]{%
	\edef\@tempa{\expandafter\noexpand\csname text#1\endcsname}
	\parbox[c]{7.5mm}{\centering{\textcolor{IslamicGreen}{\@tempa{#2}}}\\[1pt]\ttfamily\fontsize{5}{5}\selectfont{#3}}\egroup}
\newcommand{\GGX}[2][arab]{%
	\edef\@tempa{\expandafter\noexpand\csname text#1\endcsname}
	\parbox[c]{7.5mm}{\centering{\textcolor{lightgray}{\@tempa{#2}}}}\egroup}

\newcommand{\hamzaB}{\char"200D\char"0640\raise-.95ex\hbox{\char"0654}\char"200D}
%    \end{macrocode}
%</dtx-style>
%\fi
%
%
% \typeout{*************************************************************}
% \typeout{*}
% \typeout{* To finish the installation you have to move the following}
% \typeout{* file into a directory searched by XeTeX:}
% \typeout{*}
% \typeout{* \space\space\space arabxetex.sty}
% \typeout{*}
% \typeout{* and move all *.tec and *.map files to}
% \typeout{* .../fonts/misc/xetex/fontmapping/arabxetex/}
% \typeout{*************************************************************}
%
\endinput

